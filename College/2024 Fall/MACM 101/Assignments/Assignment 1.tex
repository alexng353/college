\documentclass[12pt]{article}
\usepackage{amsmath}
\usepackage{amssymb}
\usepackage{amsthm}
\usepackage{amsfonts}
\usepackage{graphicx}
\usepackage{textcomp}
\usepackage{hyperref}
\usepackage{tikz}
\usepackage{enumitem}
\usepackage{mathtools}
\usepackage{float}
\usepackage{cleveref}
\usepackage{hyperref}
\usepackage{csquotes}

\begin{document}

\title{MACM 101 Chapter 1 Homework}
\author{Alexander Ng}
\date{Sunday, September 15, 2024}

\maketitle

\section{Section 1.1}

\subsection*{Question 22}
\begin{itemize}
    \item[a)] Inclusive or, the requirement is experience with one or the other,
      and having both would still satisfy the requirement.
    \item[b)] Exclusive or, Lunch will come with either soup or salad, not both.
    \item[c)] Inclusive or, having both documents will not get you turned away.
    \item[d)] Exclusive or, publishing prevents perishing.
\end{itemize}

\subsection*{Question 24}
\begin{itemize}
    \item[a)] If you get promoted, then you have washed the Boss's car.
    \item[b)] If there are winds from the south, then there is a spring thaw.
    \item[c)] If you bought the computer less than a year ago, then the warranty
      is good.
    \item[d)] If Willy cheats, then he gets caught.
    \item[e)] If you can access the website, then you have paid a subscription
      fee.
    \item[f)] If you know the right people, then you get elected.
    \item[g)] If Carol is on a boat, then she gets seasick.
\end{itemize}

\subsection*{Question 26}
\begin{itemize}
    \item[a)] If you send me an e-mail message, then I will remember to send you
      the address.
    \item[b)] If you were born in the United States, then you are a citizen of
      this country.
    \item[c)] If you keep your textbook, then it will be a useful reference in
      your future courses.
    \item[d)] If their goalie plays well, then the Red Wings will win the 
      Stanley Cup.
    \item[e)] If you get the job, then you had the best credentials.
    \item[f)] If there is a storm, then the beach erodes.
    \item[g)] If you log on to the server, then you have a valid password.
    \item[h)] If you do not begin your climb too late, then you will reach the 
      summit.
    \item[i)] If you are among the first 100 customers tomorrow, then you will 
      get a free ice cream cone.
\end{itemize}

\subsection*{Question 38}

\begin{itemize}
    \item[a)] $(p \lor q) \lor r$

\begin{tabular}{|c|c|c|c|c|}
    \hline
    $p$ & $q$ & $r$ & $p \lor q$ & $(p \lor q) \lor r$ \\
    \hline
    0 & 0 & 0 & 0 & 0 \\
    0 & 0 & 1 & 0 & 1 \\
    0 & 1 & 0 & 1 & 1 \\
    0 & 1 & 1 & 1 & 1 \\
    1 & 0 & 0 & 1 & 1 \\
    1 & 0 & 1 & 1 & 1 \\
    1 & 1 & 0 & 1 & 1 \\
    1 & 1 & 1 & 1 & 1 \\
    \hline
\end{tabular}

    \item[b)] $(p \lor q) \land r$

\begin{tabular}{|c|c|c|c|c|}
    \hline
    $p$ & $q$ & $r$ & $p \lor q$ & $(p \lor q) \land r$ \\
    \hline
    0 & 0 & 0 & 0 & 0 \\
    0 & 0 & 1 & 0 & 0 \\
    0 & 1 & 0 & 1 & 0 \\
    0 & 1 & 1 & 1 & 1 \\
    1 & 0 & 0 & 1 & 0 \\
    1 & 0 & 1 & 1 & 1 \\
    1 & 1 & 0 & 1 & 0 \\
    1 & 1 & 1 & 1 & 1 \\
    \hline
\end{tabular}

    \item[c)] $(p \land q) \lor r$

\begin{tabular}{|c|c|c|c|c|}
    \hline
    $p$ & $q$ & $r$ & $p \land q$ & $(p \land q) \lor r$ \\
    \hline
    0 & 0 & 0 & 0 & 0 \\
    0 & 0 & 1 & 0 & 1 \\
    0 & 1 & 0 & 0 & 0 \\
    0 & 1 & 1 & 0 & 1 \\
    1 & 0 & 0 & 0 & 0 \\
    1 & 0 & 1 & 0 & 1 \\
    1 & 1 & 0 & 1 & 1 \\
    1 & 1 & 1 & 1 & 1 \\
    \hline
\end{tabular}

    \item[d)] $(p \land q) \lor r$

\begin{tabular}{|c|c|c|c|c|}
    \hline
    $p$ & $q$ & $r$ & $p \land q$ & $(p \land q) \lor r$ \\
    \hline
    0 & 0 & 0 & 0 & 0 \\
    0 & 0 & 1 & 0 & 1 \\
    0 & 1 & 0 & 0 & 0 \\
    0 & 1 & 1 & 0 & 1 \\
    1 & 0 & 0 & 0 & 0 \\
    1 & 0 & 1 & 0 & 1 \\
    1 & 1 & 0 & 1 & 1 \\
    1 & 1 & 1 & 1 & 1 \\
    \hline
\end{tabular}

    \item[e)] $(p \lor q) \land \neg r$

\begin{tabular}{|c|c|c|c|c|c|}
    \hline
    $p$ & $q$ & $r$ & $\neg r$ & $p \lor q$ & $(p \lor q) \land \neg r$ \\
    \hline
    0 & 0 & 0 & 1 & 0 & 0 \\
    0 & 0 & 1 & 0 & 0 & 0 \\
    0 & 1 & 0 & 1 & 1 & 1 \\
    0 & 1 & 1 & 0 & 1 & 0 \\
    1 & 0 & 0 & 1 & 1 & 1 \\
    1 & 0 & 1 & 0 & 1 & 0 \\
    1 & 1 & 0 & 1 & 1 & 1 \\
    1 & 1 & 1 & 0 & 1 & 0 \\
    \hline
\end{tabular}

    \item[f)] $(p \land q) \lor \neg r$

\begin{tabular}{|c|c|c|c|c|c|}
    \hline
    $p$ & $q$ & $r$ & $\neg r$ & $p \land q$ & $(p \land q) \lor \neg r$ \\
    \hline
    0 & 0 & 0 & 1 & 0 & 1 \\
    0 & 0 & 1 & 0 & 0 & 0 \\
    0 & 1 & 0 & 1 & 0 & 1 \\
    0 & 1 & 1 & 0 & 0 & 0 \\
    1 & 0 & 0 & 1 & 0 & 1 \\
    1 & 0 & 1 & 0 & 0 & 0 \\
    1 & 1 & 0 & 1 & 1 & 1 \\
    1 & 1 & 1 & 0 & 1 & 1 \\
    \hline
\end{tabular}
\end{itemize}

\subsection*{Question 42}
Given $(p \lor \neg q) \land (q \lor \neg r) \land (r \lor \neg p)$, show that
the statement is true if and only if $p$, $q$ and $r$ all have the same truth
value.

\subsubsection*{$p$, $q$, $r$ have the same truth value}

\begin{enumerate}
  \item Let $p=q=r$
  \item $(p \lor \neg q) \land (q \lor \neg r) \land (r \lor \neg p)$
  \item $(p \lor \neg p) \land (p \lor \neg p) \land (p \lor \neg p)$
  \item $(p \lor \neg p) \equiv \mathbb{T} $
  \item Apply the Idempotent Law, $p \land p \equiv p$
  \item $\mathbb{T} \land \mathbb{T} \land \mathbb{T} \equiv \mathbb{T}$
  \item $\therefore \mathbb{T}$.
\end{enumerate}

\subsubsection*{$p$, $q$, $r$ do not have the same truth value}

\begin{enumerate}
  \item Let $p=q=\neg r$\\
  It follows that $r = \neg p$
  \item $(p \lor \neg q) \land (q \lor \neg r) \land (r \lor \neg p)$
  \item $(p \lor \neg p) \land (p \lor \neg \neg p) \land (\neg p \lor \neg p)$ 
  \item Apply the Double Negation Law \\
    $(p \lor \neg p) \land (p \lor p) \land (\neg p \lor \neg p)$
  \item Apply the Idempotent Law \\
    $(p \lor \neg p) \land p \land \neg p$
  \item $p \lor \neg p \equiv \mathbb{T} $\\
    $(\mathbb{T}) \land p \land \neg p$
  \item $p \land \neg p \equiv \mathbb{F} $ \\
    $\mathbb{T} \land \mathbb{F} \equiv \mathbb{F}$
  \item $\therefore \mathbb{F} $
\end{enumerate}

% \begin{enumerate}
%   \item $(p \lor \neg q) \land (q \lor \neg r) \land (r \lor \neg p)$ \quad Original Premise
%   \item $((p \lor \neg q) \land (q \lor \neg r)) \land (r \lor \neg p)$ \quad Trivial
%   \item $((q \land (p \lor \neg q)) \lor (\neg r \land (p \lor \neg q)))
%     \land (r \lor \neg p)$ \quad Distributive Property
%   \item 
%     $[r \land ((q \land (p \lor \neg q)) \lor (\neg r \land (p \lor \neg q)))] \lor 
%     [\neg p \land ((q \land (p \lor \neg q)) \lor (\neg r \land (p \lor \neg q)))]$
%     \quad Distributive Property
%   \item Take $q \land (p \lor \neg q)$ and apply the distributive property, \\
%     $(q \land p) \lor (q \land \neg q)$. Notice $q \land \neg q$ is a
%     contradiction, and $p \lor \mathbf{F} \Leftrightarrow p$. \\
%     $\therefore q \land (p \lor \neg q) \Leftrightarrow (q \land p)$ \\
%     Rewritten,
%
%     $[r \land ((p \land q) \lor (\neg r \land (p \lor \neg q)))] \lor 
%     [\neg p \land ((p \land q) \lor (\neg r \land (p \lor \neg q)))]$
%   \item Again, apply the distributive property. \\
%     $[(r \land (p \land q)) \lor (r \land (\neg r \land p) \lor (\neg r \lor \neg q)))]
%     \lor [(\neg p \land (p \land q)) \lor (\neg p \land ((\neg r \land p) \lor (\neg r \land \neg q)))]$
%   \item Notice how in the above statement, we have \\
%     $r \land ((\neg r \land p) \lor (\neg r \land \neg q))$, which distributes to \\
%     $r \land (\neg r \land p) \lor r \land (\neg r \land \neg q)$.
%     This can easily be manipulated using the associative laws to show \\
%     $((r \land \neg r) \land p) \lor ((r \land \neg r) \land \neg q)$. $\to p \land \neg p \equiv \mathbf{F}$\\
%     $(\mathbf{F} \land p) \lor (\mathbf{F} \land \neg q)$. We know that $\mathbf{F} \land p \equiv \mathbf{F}$.\\
%     $\therefore r \land ((\neg r \land p) \lor (\neg r \land \neg q)) \equiv \mathbf{F}$
%   \item $[(r \land (p \land q)) \lor \mathbf{F})]
%     \lor [(\neg p \land (p \land q)) \lor (\neg p \land ((\neg r \land p) \lor (\neg r \land \neg q)))]$\\
%     Doing the same with $(\neg p \land ((\neg r \land p) \lor (\neg r \land \neg q)))$, we get \\
%     $(\neg p \land (\neg r \land p)) \lor (\neg p \land (\neg r \land \neg q))$\\
%     $\mathbf{F} \lor (\neg p \land (\neg r \land \neg q))$
%   \item $[(r \land (p \land q)) \lor \mathbf{F})] \lor 
%     [\mathbf{F} \lor (\neg p \land (\neg r \land \neg q))]$\\
%     Because $p \lor \mathbf{F} \equiv p$,
%   \item $(r \land (p \land q)) \lor (\neg p \land (\neg q \land \neg r))$
%   \item $\therefore (p \lor \neg q) \land (q \lor \neg r) \land (r \lor \neg p) \equiv
%     (r \land (p \land q)) \lor (\neg p \land (\neg q \land \neg r))$
% \end{enumerate}
%
% \subsubsection*{Proof}
% Show that the statement is true if and only if $p$, $q$, and $r$ all have the 
% same truth value.
%
%
% \begin{enumerate}
%   \item Assume $p=q=r$. Then,
%   \item $(r \land (p \land q)) \lor (\neg p \land (\neg q \land \neg r)$
%   \item $(p \land p \land p) \lor (\neg p \land \neg p \land \neg p)$ \\
%     By the Idempotent laws, $p \land p \equiv \mathbf{T}$,
%   \item $p \lor \neg p$.\\
%     $p \lor \neg p \equiv \mathbf{T}$
%   \item $\therefore$ when all values of $p$, $q$ and $r$ share the same truth value,
%     the statement is true.
% \end{enumerate}
%
% \begin{enumerate}
%   \item Assume $p=q$; $r=\neg p$
%   \item $(r \land (p \land q)) \lor (\neg p \land (\neg q \land \neg r)$
%   \item $(p \land p \land \neg p) \lor (\neg p \land \neg p \land \neg \neg p)$ \\
%     By the Double Negation and Idempotent Laws,
%   \item $(p \land \neg p) \lor (\neg p \land p)$ \\
%     Because $p \land \neg p$ is a Contradiction,
%   \item $\mathbf{F} \lor \mathbf{F} \equiv \mathbf{F}$.
%   \item $\therefore$ when the values of $p$, $q$ and $r$ are not all the same,
%     the statement is false.
% \end{enumerate}
%
% Only after spending hours on this, I realize that I could have taken the
% original statement, applied the above two tests to it and proved the same
% thing. I am clearly, unequivocally, even, an idiot.

\end{document}

