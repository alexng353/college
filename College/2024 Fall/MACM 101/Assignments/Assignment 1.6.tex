\documentclass[12pt]{article}
\usepackage{amsmath}
\usepackage{amssymb}
\usepackage{amsthm}
\usepackage{amsfonts}
\usepackage{graphicx}
\usepackage{textcomp}
\usepackage{hyperref}
\usepackage{tikz}
\usepackage{enumitem}
\usepackage{mathtools}
\usepackage{float}
\usepackage{cleveref}
\usepackage{hyperref}
\usepackage{csquotes}

\begin{document}

\title{MACM 101 Chapter 1.6 Homework}
\author{Alexander Ng}
\date{Tuesday, October 1, 2024}

\maketitle

\subsection*{Question 2}

This is Modus Tollens, and Modus Tollens is valid. Therefore, the conclusion of
the argument is true, given the hypotheses are also true. 

\subsection*{Question 6}

Use rules of inference to show that the hypotheses “If it does not rain or if
it is not foggy, then the sailing race will be held and the lifesaving 
demonstration will go on,” “If the sailing race is held, then the trophy will 
be awarded,” and “The trophy was not awarded” imply the conclusion “It rained.”

Let $p$ (r) be the proposition \enquote*{It rains.}

Let $q$ (f) be the proposition \enquote*{It is foggy.}

Let $r$ (s) be the proposition \enquote*{The sailing race will be held.}

Let $s$ (l) be the proposition \enquote*{The life saving demonstration will go on.}

Let $t$ (t) be the proposition \enquote*{The trophy will be awarded}.

We are given $(\neg p \lor \neg q) \to (r \land s)$, $r \to t$ and $\neg t$

\begin{enumerate}
  \item $\neg t$ (Hypothesis)
  \item $r \to t$ (Hypothesis)
  \item $\neg r$ (Modus Tollens of Step 1 and 2)
  \item $(\neg  p \lor  \neg  q) \to (r \land s)$ (Hypothesis)
  \item $(\neg (r \land s)) \to \neg (\neg p \lor \neg q))$ (Contrapositive of Step 4)
  \item $(\neg r \lor \neg s) -> (p \land q)$ (De Morgan's and Double Negative)
  \item $\neg r \lor  \neg s$ (Following Step 3)
  \item $p \land q$ (Modus Ponens of Step 6 and 7)
  \item $p$ (Simplification of Step 8)
\end{enumerate}

\subsection*{Question 12}

We are given:

\begin{enumerate}
  \item $(p \land t) \to (r \lor s)$
  \item $q \to (u \land t)$
  \item $u \to p$
  \item $\neg s$
  \item $q$
\end{enumerate}

We want to show $r$.

\begin{enumerate}
  \item $q \to (u \land t)$ (Hypothesis)
  \item $q$ (Hypothesis)
  \item $u \land t$ (Modus Ponens of 1 and 2)
  \item $u$ (Simplification of 3)
  \item $t$ (Simplification of 3)
  \item $u \to p$ (Hypothesis)
  \item $p$ (Modus Ponens of 4 and 5)
  \item $(p \land t) \to (r \lor s)$
  \item $r \lor s$ (Modus Ponens of 8, 7 and 5)
  \item $\neg s$ (Hypothesis)
  \item $r$ (Disjunctive Syllogism of 9 and 10)
\end{enumerate}

\subsection*{Question 18}

Some $s$ exists such that $S(s, \text{Max} )$, but we cannot conclude that Max
is one such $s$ from the givens. Therefore, it does not follow that 
$S(\text{Max} , \text{Max} )$, and the first step is invalid.

\subsection*{Question 20}

\begin{enumerate}[label=(\alph*)]
  \item Invalid. $(x > 0) \to (x^2 > 0)$ does not mean $(x^2 > 0) \to (x>0)$
\item Valid (Modus Ponens)
\end{enumerate}

\subsection*{Question 28}

I ran out of time trying to solve this problem

\end{document}
