\documentclass[12pt]{article}
\usepackage{amsmath}
\usepackage{amssymb}
\usepackage{amsthm}
\usepackage{amsfonts}
\usepackage{graphicx}
\usepackage{textcomp}
\usepackage{hyperref}
\usepackage{tikz}
\usepackage{enumitem}
\usepackage{mathtools}
\usepackage{float}
\usepackage{cleveref}
\usepackage{hyperref}
\usepackage{csquotes}
\usepackage{multicol}
\usepackage{booktabs}

\begin{document}

\title{MACM 101 Chapter 2.1 Homework}
\author{Alexander Ng}
\date{Monday, October 7, 2024}

\maketitle

\subsection*{Question 12}

\begin{enumerate}[label=(\alph*)]
  \item Given $\emptyset \in \{ \emptyset \}$.\\
    $\{ \emptyset \}$ is a set containing only the $\emptyset$, therefore,
    the given statement is \textbf{True}.
  \item Given $\emptyset \in \{\emptyset , \{ \emptyset \}\}$.\\
    $\emptyset$ is an element of $\{\emptyset , \{ \emptyset \}\}$, therefore,
    the given statement is \textbf{True}.
  \item Given $\{ \emptyset \} \in \{\emptyset \}$.\\
    $\{ \emptyset \}$ is not an element of $\{\emptyset \}$, therefore, the given
    statement is \textbf{False}.
  \item Given $\{ \emptyset \} \in \{ \{ \emptyset \}\}$.\\
    $\{ \emptyset \}$ is an element of $\{ \{ \emptyset \}\}$, therefore,
    the given statement is \textbf{True}.
  \item Given $\{\emptyset\} \subset \{\emptyset,\{ \emptyset \}\}$.\\
    Every element in $\{\emptyset\}$ is also an element of 
    $\{ \emptyset, \{ \emptyset \} \}$, therefore the given statement is \textbf{True}.
  \item Given $\{ \{ \emptyset \} \} \subset \{ \emptyset, \{ \emptyset \}\}$.\\
    Every element in $\{ \{ \emptyset \} \}$ is also an element of 
    $\{ \emptyset, \{ \emptyset \} \}$, therefore the given statement is \textbf{True}.
  \item Given $\{ \{ \emptyset \} \} \subset \{ \{ \emptyset \}, \{ \emptyset \} \}$. \\
    Every element in $\{ \{ \emptyset \} \}$ is also an element of
    $\{ \{ \emptyset \}, \{ \emptyset \} \}$. However, the sets are equal,
    therefore the given statement is \textbf{False}.
\end{enumerate}

\subsection*{Question 34}

% 34. Let A = {a, b, c}, B = {x, y}, and C = {0, 1}. Find
% a) A × B × C. b) C × B × A.
% c) C × A × B. d) B × B × B

Let $A = \{ a, b, c \}$, $B = \{ x, y \}$ and $C = \{ 0, 1 \}$.

\begin{enumerate}[label=(\alph*)]
  \item $A \times B \times C$ \\
    $= (A \times B) \times C$ \\
    $= \{(a, x), (a, y), (b, x), (b, y), (c, x), (c, y)\} \times C$ \\
    $= \{(a, x, 0), (a, x, 1), (a, y, 0), (a, y, 1), (b, x, 0), (b, x, 1), \\
    (b, y, 0), (b, y, 1), (c, x, 0), (c, x, 1), (c, y, 0), (c, y, 1)\}$
  \item $C \times B \times A$ \\
    $= (C \times B) \times A$ \\
    $= \{(0, x), (0, y), (1, x), (1, y)\} \times A$ \\
    $= \{(0, x, a), (0, x, b), (0, x, c), (0, y, a), (0, y, b), (0, y, c), \\
    (1, x, a), (1, x, b), (1, x, c), (1, y, a), (1, y, b), (1, y, c)\}$
  \item $C \times A \times B$ \\
    $= (C \times A) \times B$ \\
    $= \{(0, a), (0, b), (0, c), (1, a), (1, b), (1, c)\} \times B$ \\
    $= \{(0, a, x), (0, a, y), (0, b, x), (0, b, y), (0, c, x), (0, c, y) \\
    (1, a, x), (1, a, y), (1, b, x), (1, b, y), (1, c, x), (1, c, y)\}$
  \item $B \times B \times B$ \\
    $= (B \times B) \times B$ \\
    $= \{(x, x), (x, y), (y, x), (y, y)\} \times B$ \\
    $= \{(x, x, x), (x, x, y), (x, y, x), (x, y, y), (y, x, x), (y, x, y),  
    (y, y, x), (y, y, y)\}$
\end{enumerate}

\subsection*{Question 44}

Prove or disprove that if $A$, $B$, and $C$ are nonempty sets and
$A \times B = A \times C$, then $B = C$.

Assume $B \neq C$. 

This means $\exists x (x \in B \land x \notin C)$

However, because $A \times B$ is defined as $\{(a, b) | a \in A \land b \in B\}$,
and $A \times C$ is defined as $\{(a, c) | a \in A \land c \in B\}$,
if $\exists x (x \in B \land x \notin C)$, this implies 
$\exists x ((a, x) \in A \times B \land (a, x) \notin A \times C)$

$\therefore A \times B \neq A \times C$

This is a contradiction, thus $B = C$.

$\qed$

\subsection*{Question 50}

% This exercise presents Russell’s paradox. Let S be the
% set that contains a set x if the set x does not belong to
% itself, so that S = {x ∣ x ∉ x}.
% a) Show the assumption that S is a member of S leads to Links a contradiction.
% b) Show the assumption that S is not a member of S leads
% to a contradiction.
% By parts (a) and (b) it follows that the set S cannot be defined as it was. This paradox can be avoided by restricting
% the types of elements that sets can have.

This exercise presents Russell's paradox. Let $S$ be the set that contains a
set $x$ if the set $x$ does not belong to itself, so that 
$S = \{ x \mid x \notin x \}$.

\subsubsection*{Part a}

Show the assumption that $S$ is a member of $S$ leads to Links a contradiction.

Assume $S \in S$.

By definition of $S$, this means $S \notin S$.

$S \in S$ implies $S \notin S$ is a contradiction.

\subsubsection*{Part b}

Show the assumption that $S$ is not a member of $S$ leads to a contradiction.

Assume $S \notin S$.

By definition of $S$, this means $S \in S$.

$S \notin S$ implies $S \in S$ is a contradiction.

\end{document}
