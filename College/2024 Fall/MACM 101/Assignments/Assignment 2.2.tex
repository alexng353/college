\documentclass[12pt]{article}
\usepackage{amsmath}
\usepackage{amssymb}
\usepackage{amsthm}
\usepackage{amsfonts}
\usepackage{graphicx}
\usepackage{textcomp}
\usepackage{hyperref}
\usepackage{tikz}
\usepackage{enumitem}
\usepackage{mathtools}
\usepackage{float}
\usepackage{cleveref}
\usepackage{hyperref}
\usepackage{csquotes}
\usepackage{multicol}
\usepackage{booktabs}

\begin{document}

\title{MACM 101 Chapter 2.2 Homework}
\author{Alexander Ng}
\date{Monday, October 7, 2024}

\maketitle

\subsection*{Question 6}

\subsubsection*{Part a}

$A \cup \emptyset = A$

$A \cup B = \{ x \left| x \in A \lor x \in B \right.\}$

By definition, $A \cup \emptyset = \{ x \left| x \in A \lor x \in \emptyset \right.\}$

The empty set does not contain any elements, so $x \notin \emptyset$

$A \cup \emptyset = \{ x \left| x \in A \lor \mathbf{F} \right.\}$

By the Identity Laws of Propositional Logic, $p \lor \mathbf{F} = p$

$A \cup \emptyset = \{ x \left| x \in A \right.\}$

$\therefore A \cup \emptyset = A$

\subsubsection*{Part b}

$A \cap U = A$

$A \cap B = \{ x \left| x \in A \land x \in B \right.\}$

By definition, $A \cap U = \{ x \left| x \in A \land x \in U \right.\}$

Because $U$ is the universal set, $x \in U \equiv \mathbf{T}$

So, $A \cap U = \{ x \left| x \in A \land \mathbf{T} \right.\}$

By the Identity Laws of Propositional Logic, $p \land \mathbf{T} = p$

$A \cap U = \{ x \left| x \in A \right.\}$

$\therefore A \cap U = A$

\subsection*{Question 8}

% Prove the idempotent laws in Table 1 by showing that
% a) A ∪ A= A.
% b) A ∩ A= A.

\subsubsection*{Part a}

$A \cup A = A$

By definition, $A \cup A = \{ x \left| x \in A \lor x \in A \right.\}$

The left and right side of the disjunction are the same set, so by the 
Idempotent Law of Propositional Logic, $p \lor p \equiv p$,

$x \in A \lor x \in A \equiv x \in A$

$A \cup A = \{ x \left| x \in A \right.\}$

$\therefore A \cup A = A$

\subsubsection*{Part b}

$A \cap A = A$

By definition, $A \cap A = \{ x \left| x \in A \land x \in A \right.\}$

The left and right side of the conjunction are the same set, so by the 
Idempotent Law of Propositional Logic, $p \land p \equiv p$,

$x \in A \land x \in A \equiv x \in A$

$A \cap A = \{ x \left| x \in A \right.\}$

$\therefore A \cap A = A$

\subsection*{Question 10}

\subsubsection*{Part a}

$A - \emptyset = A$

$= A \cap \overline{\emptyset}$

The complement of the empty set is the universal set, so $\overline{\emptyset} = U$

By Question 6, Part b, $A \cap U = A$

$= A \cap U = A$

$\therefore A - \emptyset = A$

\subsubsection*{Part b}

$\emptyset - A = \emptyset$

By definition, 

$= \emptyset \cap \overline{A}$

By definition,

$= \{ x \left| x \in \emptyset \land x \in \overline{A} \right.\}$

Because $x \in \emptyset \equiv \mathbf{F}$

$= \{ x \left| \mathbf{F} \land x \in \overline{A} \right.\}$

By the Domination Laws of Propositional Logic, $p \land \mathbf{F} = \mathbf{F}$

$= \{ x \left| \mathbf{F} \right.\}$

$\therefore \emptyset - A = \emptyset$

\end{document}
