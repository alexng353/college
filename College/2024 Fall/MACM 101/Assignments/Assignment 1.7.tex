\documentclass[12pt]{article}
\usepackage{amsmath}
\usepackage{amssymb}
\usepackage{amsthm}
\usepackage{amsfonts}
\usepackage{graphicx}
\usepackage{textcomp}
\usepackage{hyperref}
\usepackage{tikz}
\usepackage{enumitem}
\usepackage{mathtools}
\usepackage{float}
\usepackage{cleveref}
\usepackage{hyperref}
\usepackage{csquotes}

\begin{document}

\title{MACM 101 Chapter 1.7 Homework}
\author{Alexander Ng}
\date{Tuesday, October 1, 2024}

\maketitle

\subsection*{Question 2}

Use a direct proof to show that the sum of two even integers is even.

Suppose $a$ and $b$ are even integers. There exists integers $c$ and $d$ such that
$a = 2c$ and $b = 2d$. Adding, we obtain that $a + b = 2c + 2d = 2(c + d)$.
$a + b$ is $2$ times the integer $c + d$. Therefore, $a + b$ is even.

\subsection*{Question 4}

Show that the additive inverse, or negative, of an even number is an even number
using a direct proof.

Suppose $a$ is an even integer. Then there exists an integer $s$ such that 
$a = 2s$. The additive inverse of $a$ is $-a = -2s$, which is equal to $2(-s)$.
Since this is $2$ times the integer $-s$, the additive inverse of $a$ is even.

\subsection*{Question 8}

Prove that if $n$ is a perfect square, then $n + 2$ is not a perfect square.

Let $n = m^2$. If $m = 0$, then $n+2=2$, which is not a perfect square. This
means $m \geq 1$. The smallest perfect square greater than $n$ is $(m+1)^2$.
We have $(m+1)^2 = m^2 + 2m + 1 = n + 2m + 1$.

$n + 2m + 1 > n + 2 \cdot 1 + 1 > n+2$. Therefore, $n+2$ cannot be a perfect 
square.

\subsection*{Question 14}

Prove that if $x$ is rational and $x \ne 0$, then $\frac{1}{x}$ is rational.

If $x$ is rational, then $x$ can be written as a fraction of two integers, $a$
and $b$. If $x = \frac{a}{b}$, then $\frac{1}{x} = \frac{b}{a}$, where $b$ and
$a$ are both integers. Therefore, $\frac{1}{x}$ is also rational.

\subsection*{Question 24}

Show that if you pick three socks from a drawer containing just blue socks and
black socks, you must get either a pair of blue socks or a pair of black socks.

Proof by contradiction:

Suppose we don't draw a pair of blue socks or a pair of black socks. Then we
drew at most one blue sock and one black sock. However, one of each socks is
only two socks. But we are drawing three socks. Therefore our supposition that
we did not get a pair of blue socks or a pair of black socks is incorrect.
Therefore, we must have drawn a pair of blue socks or a pair of black socks.

\subsection*{Question 34}

Suppose $x = \frac{a}{b}$ where $a$ and $b$ are integers with $b \ne 0$.
Then, $\frac{x}{2} = \frac{a}{2b}$, which is rational since $a$ and $2b$ are
both integers with $2b \ne 0$.

Suppose $\frac{x}{2} = \frac{a}{b}$ where $a$ and $b$ are integers with $b \ne 0$.
Then $x = \frac{2a}{b}$, and $3x-1 = \frac{6a}{b}-1 = \frac{6a-b}{b}$. This is
rational since $6a-b$ and $b$ are integers with $b \ne 0$.

Suppose that $3x-1 = \frac{a}{b}$ where $a$ and $b$ are integers with $b \ne 0$.
Then $x = \frac{\frac{a}{b}+1}{3} = \frac{a+b}{3b}$ and this is rational, since
$a+b$ and $3b$ are both integers with $3b \ne 0$.

This shows that $(i)$ implies $(ii)$, $(ii)$ implies $(iii)$ and $(iii)$
implies $(i)$, which is sufficient to show that these three statements are
equivalent.

\subsection*{Question 40}

Counterexample: $15$.

$9 + 9 + n = 18 + n > 15$, therefore we can only use one $9$.

$9 + 4 + 4 = 17 > 15$, therefore we can use at most one $9$ and one $4$.

$9 + 4 + 1 = 14$ is the greatest sum we can create with at most one $9$ and 
one $4$.

Clearly, at least one more $1$ is required to create $15$. Therefore, it is
impossible to write the number $15$ as the sum of the squares of three integers. 

\subsection*{Question 44}

Ran out of time

\end{document}
