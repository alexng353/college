\documentclass[12pt]{article}
\usepackage{amsmath}
\usepackage{amssymb}
\usepackage{amsthm}
\usepackage{amsfonts}
\usepackage{graphicx}
\usepackage{textcomp}
\usepackage{hyperref}
\usepackage{tikz}
\usepackage{enumitem}
\usepackage{mathtools}
\usepackage{float}
\usepackage{cleveref}
\usepackage{hyperref}
\usepackage{csquotes}

\begin{document}

\title{MACM 101 Chapter 1 Homework}
\author{Alexander Ng}
\date{Monday, September 30, 2024}

\maketitle

\subsection*{Question 8}

\subsubsection*{Part a}

Every rabbit hops.

\subsubsection*{Part b}

Every animal is a rabbit and it hops.

\subsubsection*{Part c}

There exists an animal such that, if it is a rabbit, then it hops.

\subsubsection*{Part d}

There exists an animal that is a rabbit and it hops.

\subsection*{Question 20}

\subsubsection*{Part a}

\[
P(-5) \lor P(-3) \lor P(-1) \lor P(1) \lor P(3) \lor P(5)
\]

\subsubsection*{Part b}

\[
P(-5) \land P(-3) \land P(-1) \land P(1) \land P(3) \land P(5)
\]

\subsubsection*{Part c}

\[
P(-5) \land P(-3) \land P(-1) \land P(3) \land P(5)
\]

\subsubsection*{Part d}

\[
P(1) \lor P(3) \lor P(5)
\]

\subsubsection*{Part e}

\[
(\neg P(-5) \lor \neg P(-3) \lor \neg P(-1) \lor \neg P(1) \lor \neg P(3)
\lor \neg P(5)) \land (P(-5) \land P(-3) \land P(-1))
\]

\subsection*{Question 32}

\subsubsection*{Part a}

\enquote{All dogs have fleas}

Let the domain be dogs and $P(x)$ mean \enquote{$x$ has fleas}

$\forall x P(x)$

Negate, then apply De Morgan's Laws for Quantifiers

$\neg (\forall x P(x)) \equiv \exists x \neg P(x)$

\enquote{There is a dog that does not have fleas}

\subsubsection*{Part b}

\enquote{There is a horse that can add}

Let the domain be horses and $P(x)$ mean \enquote{$x$ can add}

$\exists x P(x)$

Negate, then appy De Morgan's Laws.

$\neg (\exists x P(x)) \equiv \forall x \neg P(x)$

\subsubsection*{Part c}

\enquote{Every koala can climb}

Let the domain be koalas and $P(x)$ mean \enquote{$x$ can climb}

$\forall x P(x)$

Negate, then apply De Morgan's Laws

$\neg (\forall x P(x)) \equiv \exists x \neg P(x)$

\enquote{There exists a koala that cannot climb}

\subsubsection*{Part d}

\enquote{No monkey can speak french}

Let the domain be monkeys and $P(x)$ mean \enquote{$x$ can speak french}

$\forall x \neg P(x)$

Negate, then apply De Morgan's Law and the Double Negation Law

$\neg (\forall x \neg P(x)) \equiv \exists x \neg (\neg P(x)) \equiv \exists x P(x)$

\subsubsection*{Part e}

\enquote{There exists a pig that can swim and catch fish}

Let the domain be pigs, $P(x)$ mean \enquote{$x$ can swim} and $Q(x)$ mean 
\enquote{$x$ can catch fish}

$\exists x (P(x) \land Q(x))$

Negate, then apply De Morgan's Law for Quantifiers and the regular De Morgan's Law
$\neg (\exists x (P(x) \land Q(x))) \equiv \forall x \neg (P(x) \land Q(x)) 
\equiv \forall x (\neg P(x) \lor \neg Q(x))$

\enquote{All pigs can not swim or not catch fish.}

\subsection*{Question 36}

\subsubsection*{Part a}

\[
\forall x(-2 < x < 3)
\]

\[
-2 < x < 3 \text{ implies } -2 < x \text{ and } x < 3
\]

\[
\forall x((-2 < x) \land (x < 3))
\]

Negate:

\[
\neg \forall x((-2 < x) \land (x < 3))
\]

Using De Morgan's Laws for Quantifiers:

\[
\equiv \exists x \neg ((-2 < x) \land (x < 3))
\]

Using De Morgan's Laws:

\[
\equiv \exists x ((\neg (-2 < x)) \lor (\neg (x < 3)))
\]

When \( -2 < x \) is not true, we require \( x \leq -2 \).

When \( x < 3 \) is not true, we require \( x \geq 3 \).

\[
\equiv \exists x ((x \leq -2) \lor (x \geq 3))
\]

\subsubsection*{Part b}

\[
\forall x (0 \leq x < 5)
\]

\[
0 \leq x < 5 \text{ implies } 0 \leq x \text{ and } x < 5
\]

\[
\forall x ((0 \leq x) \land (x < 5))
\]

Negate:

\[
\neg \forall x ((0 \leq x) \land (x < 5))
\]

Using De Morgan's Laws for Quantifiers:

\[
\equiv \exists x \neg ((0 \leq x) \land (x < 5))
\]

Using De Morgan's Laws:

\[
\equiv \exists x ((\neg (0 \leq x)) \lor (\neg (x < 5)))
\]

When \( 0 \leq x \) is not true, we require \( x < 0 \).

When \( x < 5 \) is not true, we require \( x \geq 5 \).

\[
\equiv \exists x ((x < 0) \lor (x \geq 5))
\]

\subsubsection*{Part c}

\[
\exists x (-4 \leq x \leq 1)
\]

\[
-4 \leq x \leq 1 \text{ implies } -4 \leq x \text{ and } x \leq 1
\]

\[
\exists x ((-4 \leq x) \land (x \leq 1))
\]

Negate:

\[
\neg \exists x ((-4 \leq x) \land (x \leq 1))
\]

Using De Morgan's Laws for Quantifiers:

\[
\equiv \forall x \neg ((-4 \leq x) \land (x \leq 1))
\]

Using De Morgan's Laws:

\[
\equiv \forall x ((\neg (-4 \leq x)) \lor (\neg (x \leq 1)))
\]

When \( -4 \leq x \) is not true, we require \( x < -4 \).

When \( x \leq 1 \) is not true, we require \( x > 1 \).

\[
\equiv \forall x ((x < -4) \lor (x > 1))
\]

\subsubsection*{Part d}

\[
\exists x (-5 < x < -1)
\]

\[
-5 < x < -1 \text{ implies } -5 < x \text{ and } x < -1
\]

\[
\exists x ((-5 < x) \land (x < -1))
\]

Negate:

\[
\neg \exists x ((-5 < x) \land (x < -1))
\]

Using De Morgan's Laws for Quantifiers:

\[
\equiv \forall x \neg ((-5 < x) \land (x < -1))
\]

Using De Morgan's Laws:

\[
\equiv \forall x ((\neg (-5 < x)) \lor (\neg (x < -1)))
\]

When \( -5 < x \) is not true, we require \( x \leq -5 \).

When \( x < -1 \) is not true, we require \( x \geq -1 \).

\[
\equiv \forall x ((x \leq -5) \lor (x \geq -1))
\]

\subsection*{Question 48}

\subsubsection*{Part a}

If $(\forall x P(x)) \lor A$ is true, then $A$ is true or for all values $y$ 
we have that $P(y)$ is true. Then $P(y) \lor A$ is true for all values of $y$, 
implying that $\forall x (P(x) \lor A)$ is true.

If $(\forall x P(x)) \lor A$ is false, then $A$ is false and there exists a value 
of $y$ such that $P(y)$ is false. Then $P(y) \lor A$ is false, implying that
$\forall x (P(x) \lor A)$ is also false.

Thus, the two expressions always have the same truth value and therefore they are 
logically equivalent.

\subsubsection*{Part b}

If $(\exists x P(x)) \lor A$ is true, then $A$ is true or there exists a value $y$ 
for which $P(y)$ is true. Then $P(y) \lor A$ is true, implying that
$\exists x (P(x) \lor A)$ is true.

If $(\exists x P(x)) \lor A$ is false, then $A$ is false, and for all values $y$ 
we have that $P(y)$ is false. Then $P(y) \lor A$ is false for every value of $y$, 
implying that $\exists x (P(x) \lor A)$ is also false.

Thus, the two expressions always have the same truth value and therefore they are 
logically equivalent.

\end{document}
