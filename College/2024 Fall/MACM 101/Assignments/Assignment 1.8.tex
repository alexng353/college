\documentclass[12pt]{article}
\usepackage{amsmath}
\usepackage{amssymb}
\usepackage{amsthm}
\usepackage{amsfonts}
\usepackage{graphicx}
\usepackage{textcomp}
\usepackage{hyperref}
\usepackage{tikz}
\usepackage{enumitem}
\usepackage{mathtools}
\usepackage{float}
\usepackage{cleveref}
\usepackage{hyperref}
\usepackage{csquotes}
\usepackage{multicol}
\usepackage{booktabs}

\begin{document}

\title{MACM 101 Chapter 1.8 Homework}
\author{Alexander Ng}
\date{Tuesday, October 1, 2024}

\maketitle

\subsection*{Question 4}

Prove that there are no perfect cubes less than 1000 that are the sum of the
cubes of two positive integers. 

The cubes less than 1000 are $1$, $8$, $27$, $64$, $125$, $216$, $343$, $512$
and $729$.

If we try to sum all possible combinations of two of these integers, we see
that none of them work. By exhaustion, we conclude that no cube less than 1000
is the sum of two cubes. (See next page for the exhaustive list)

\pagebreak

\begin{multicols}{2}
\noindent
\begin{tabular}{l}
       1 + 1 = 2 \\
       1 + 8 = 9 \\
     1 + 27 = 28 \\
     1 + 64 = 65 \\
   1 + 125 = 126 \\
   1 + 216 = 217 \\
   1 + 343 = 344 \\
   1 + 512 = 513 \\
   1 + 729 = 730 \\
      8 + 8 = 16 \\
     8 + 27 = 35 \\
     8 + 64 = 72 \\
   8 + 125 = 133 \\
   8 + 216 = 224 \\
   8 + 343 = 351 \\
   8 + 512 = 520 \\
   8 + 729 = 737 \\
    27 + 27 = 54 \\
    27 + 64 = 91 \\
  27 + 125 = 152 \\
  27 + 216 = 243 \\
  27 + 343 = 370 \\
  27 + 512 = 539 \\
  27 + 729 = 756 \\
   64 + 64 = 128 \\
  64 + 125 = 189 \\
  64 + 216 = 280 \\
  64 + 343 = 407 \\
  64 + 512 = 576 \\
  64 + 729 = 793 \\
 125 + 125 = 250 \\
 125 + 216 = 341 \\
 125 + 343 = 468 \\
 125 + 512 = 637 \\
 125 + 729 = 854 \\
\end{tabular}

\columnbreak

\begin{tabular}{l}
 216 + 216 = 432 \\
 216 + 343 = 559 \\
 216 + 512 = 728 \\
 216 + 729 = 945 \\
 343 + 343 = 686 \\
 343 + 512 = 855 \\
343 + 729 = 1072 \\
512 + 512 = 1024 \\
512 + 729 = 1241 \\
729 + 729 = 1458 \\
\end{tabular}
\end{multicols}

\pagebreak

\subsection*{Question 6}

$\min(a, \min(b, c)) = \min(\min(a, b), c)$

If $a$ is the smallest (or equal to the smallest), then clearly, 
$a \leq \min(b, c)$, so the left side is just $a$. On the right side, we
have $\min(\min(a, b), c) = \min(a, c) = a$. So we have $a=a$.

If $b$ is the smallest (or equal to the smallest), we can use the same
reasoning to show that $\min(a, \min(b, c)) = \min(a, b) = b$ on the left,
and $\min(\min(a, b), c) = \min(b, c) = b$ on the right, and we have $b=b$.

If $c$ is the smallest (or equal to the smallest), then we have
$\min(a, \min(b,c)) = \min(a, c) = c$ on the left and $\min(\min(a, b), c) = c$
on the right. Again, we have $c = c$.

Since one of the three has to be smallest, all cases have been taken care of.

\subsection*{Question 10}

$1$ has this property, since the only positive integer not exceeding $1$ is $1$,
and therefore the sum is $1$. Therefore, there exists a positive integer that
equals the sum of the positive integers not exceeding it.

This is a constructive proof.

\subsection*{Question 34}

\end{document}
