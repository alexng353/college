\documentclass[12pt]{article}
\usepackage{amsmath}
\usepackage{amssymb}
\usepackage{amsthm}
\usepackage{amsfonts}
\usepackage{graphicx}
\usepackage{textcomp}
\usepackage{hyperref}
\usepackage{tikz}
\usepackage{enumitem}
\usepackage{mathtools}

\begin{document}

\title{MACM 101 Lecture 1.2 - Friday September 13}
\author{Alexander Ng}
\date{September 13, 2024}

\maketitle

\section{Summary}

Valid arguments

inference rules for propositional logic

using rules of inference to build arguments

rules of inference for quantified statements

building arguments for quantified statements

\subsection{Goals}

Provide a foundation for proof theory - a purely deductive, valid argument

\section{???}

How can one prove that "Toilet paper should be installed over as opposed to
under"

\section{What is an Argument}

We are talking about a proof in a formal axiomatic system where there is no
ambiguity or probability (no application in the real world) :skull:.

We are only concerned with deductive arguments

\subsection{Truth vs Validity} % memorize me

Truth is predicated on propositions, which are either true or false

Validity is predicated on \textbf{deductive arguments}, which are either valid or invalid

\subsection{Definition}

A deductive argument is a sequence of declarative statements.

A valid deductive argument is an argument such that no matter what particular
statement are substitude for the statement 


\subsubsection{Formal Definition of a Valid Argument}

a deductive argument is valid if and only if the premises provide conclusive
proof of the conclusion

Either of the following must hold:

if the premises of a vlalid argument are all true, then its conclusion must also be true

it is impossible for the conclusion of a valid argument to be false while its premises are true

In other words, a tautology is a necessary and sufficient condition for a valid argument (???)



\end{document}
