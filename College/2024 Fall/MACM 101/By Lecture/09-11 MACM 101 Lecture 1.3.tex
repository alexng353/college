\documentclass{article}
\usepackage{amsmath}
\usepackage{amssymb}
\usepackage{array}

\begin{document}

\section{Implication}

\begin{equation}
  \neg q \implies \neg p
\end{equation}

\begin{equation}
\begin{array}{|c|c|c|c|c|c|}
\hline
p & q & \neg p & \neg q & p \implies q & \neg q \implies \neg p \\
\hline
0 & 0 & 1 & 1 & 1 & 1 \\
0 & 1 & 1 & 0 & 1 & 1 \\
1 & 0 & 0 & 1 & 0 & 0 \\
1 & 1 & 0 & 0 & 1 & 1 \\
\hline
\end{array}
\end{equation}

\[
  \therefore p \implies q \text{ is logically equivalent to } \neg q \implies \neg p
\]

So, $\neg q \implies \neg p$ is the \textbf{contrapositive} of $p \implies q$. 

$p \implies q$ \textbf{material implication}

$q \implies p$ is called the \textbf{converse} of material implication.

$(\neg q \implies \neg p)$ is called the \textbf{contrapositive} of material implication, and

$(\neg p \implies \neg q)$ is called the \textbf{inverse} of material implication.

\section{First Identity}

$p \implies q$ means $p \therefore q$. 

If the argument is valid and $p$ is true, then $q$ is also true.

If the argument is valid and $p$ is false, then we can conclude nothing about $q$.

Thus, $p \implies q$ is logically equivalent to $\neg p \lor q$.


\begin{equation}
\begin{array}{|c|c|c|c|c|}
\hline
p & q & \neg p & p \implies q & \neg p \lor q \\
\hline
0 & 0 & 1 & 1 & 1 \\
0 & 1 & 1 & 1 & 1 \\
1 & 0 & 0 & 0 & 0 \\
1 & 1 & 0 & 1 & 1 \\
\hline
\end{array}
\end{equation}


This is known as \textbf{implication reduction}.

\begin{quote}
    This is important because we \textit{always} want to remove implication from our logical statements.
\end{quote}

\section{De Morgan's Laws}

\begin{equation}
  \neg (p \land q) \equiv \neg p \lor \neg q
\end{equation}

\begin{equation}
  \neg (p \lor q) \equiv \neg p \land \neg q
\end{equation}

\textbf{Many of the Axiomatic Laws come in pairs (principle of duality)}

\begin{quote}
    Show the truth table (exercise)
\end{quote}

\textbf{Propositional Identities are given on the exam.}

\section{Principle of Duality}
\begin{quote}
    (exercise for the reader (???))
\end{quote}

\end{document}
