\documentclass[12pt]{article}
\usepackage{amsmath}
\usepackage{amssymb}
\usepackage{amsthm}
\usepackage{amsfonts}
\usepackage{graphicx}
\usepackage{textcomp}
\usepackage{hyperref}
\usepackage{tikz}
\usepackage{enumitem}
\usepackage{mathtools}
\usepackage{float}
\usepackage{cleveref}
\usepackage{hyperref}
\usepackage{csquotes}

\begin{document}

\title{MACM 101 Lecture 09-16 S1.6 - Pearce}
\author{Alexander Ng}
\date{Friday, September 13, 2024}

\maketitle

\section*{Review}

\subsection{Truth and Validity}
\begin{itemize}
  \item Truth is predicated on propositions, which are either true or false
  \item Validity is predicated on \textbf{deductive arguments}, which are either valid or invalid
\end{itemize}

\subsection{Formal Definition of a Valid Argument}

a deductive argument is valid if and only if the premises provide conclusive
proof of the conclusion

Either of the following must hold:

\begin{itemize}
  \item if the premises of a valid argument are all true, then its conclusion must also be true
  \item it is impossible for the conclusion of a valid argument to be false while its premises are true
  \item In other words, a tautology is a necessary and sufficient condition for a valid argument (???)
\end{itemize}

\subsection{Example of a Modus Ponens}

If $\sqrt{2} > \frac{3}{2}$, then $\sqrt{2}^2 > \frac{3}{2}^2$

We know that $\sqrt{2}>\frac{3}{2}$, Consequently, $2 > \frac{9}{4}$

The argument is valid because it is modus ponens, but it is not correct because
the conclusion is false.

\subsection{Modus Ponens}

\begin{align*}
  p \to q \\
  p \\
  \therefore q
\end{align*}

$\uparrow$ Please fucking research this $\uparrow$

\subsection{Modus Tollens}

\center
$p \to q, \therefore \neg q \to \neg p$
\endcenter

\section{The Rules of Inference and Valid Argumentation}

*Find table on the rules of inference from Rosen

\begin{align}
  p \to q \\
  \neg q \to \neg p\\
  \neg p \to r\\
  \neg q \to r\\
  r \to s\\
  \therefore \neg q \to s
\end{align}


\end{document}
