
\documentclass[12pt]{article}
\usepackage{amsmath}
\usepackage{amssymb}
\usepackage{amsthm}
\usepackage{amsfonts}
\usepackage{graphicx}
\usepackage{textcomp}
\usepackage{hyperref}
\usepackage{tikz}
\usepackage{enumitem}
\usepackage{mathtools}
\usepackage{float}
\usepackage{cleveref}
\usepackage{hyperref}
\usepackage{csquotes}
\usepackage{multicol}
\usepackage{booktabs}

\begin{document}

\title{Proof that there is no largest integer}
\author{Alexander Ng}
\date{October 02, 2024}

\maketitle

Let the domain be the set of integers.

Assume there is a largest integer.

"There is a largest integer" can be rewritten as:

$\exists x \forall y (x > y)$

Consider the case where $y = x + 1$. 

This leads to the statement $(x > x + 1)$, which is a contradiction.

Thus, there does not exist an integer $x$ such that for all integers $y$,
$x > y$. Therefore, there is no largest integer.

Q.E.D.

\end{document}
