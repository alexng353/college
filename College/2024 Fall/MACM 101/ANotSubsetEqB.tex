
\documentclass[12pt]{article}
\usepackage{amsmath}
\usepackage{amssymb}
\usepackage{amsthm}
\usepackage{amsfonts}
\usepackage{graphicx}
\usepackage{textcomp}
\usepackage{hyperref}
\usepackage{tikz}
\usepackage{enumitem}
\usepackage{mathtools}
\usepackage{float}
\usepackage{cleveref}
\usepackage{hyperref}
\usepackage{csquotes}

\begin{document}

\title{MACM 101 Chapter 1.4}
\author{Alexander Ng}
\date{September 18, 2024}

\maketitle

\section{A Not Subset of B}

$A \nsubseteq B = \neg(A \subseteq B)$

$= \neg \forall x \in A \implies x \in B$

$\exists \neg (x \in A \implies x \in B)$

$\exists \neg (\neg x \in A \lor x \in B)$

de morgans

$\exist x \in A \land \neg x \in B$

$\exist x \in A \land x \notin B$

\section{$\emptyset$ is a subset of every set}

The empty set is a subset of $A$ unless there is some elemnt in $\emptyset$ 
that is not in $A$.

So if $\emptyset$ is not a subset of $A$ then there is an element in $\emptyset$.

But, $\emptyset$ has no elements and hence this is a contradiction, so the
$\emptyset$ must be a subset of $A$

\end{document}
