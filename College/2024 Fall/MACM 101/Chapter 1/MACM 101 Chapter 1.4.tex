\documentclass[12pt]{article}
\usepackage{amsmath}
\usepackage{amssymb}
\usepackage{amsthm}
\usepackage{amsfonts}
\usepackage{graphicx}
\usepackage{textcomp}
\usepackage{hyperref}
\usepackage{tikz}
\usepackage{enumitem}
\usepackage{mathtools}
\usepackage{float}
\usepackage{cleveref}
\usepackage{hyperref}
\usepackage{csquotes}

\begin{document}

\title{MACM 101 Chapter 1.4}
\author{Alexander Ng}
\date{September 18, 2024}

\maketitle

\section{Quantifiers}

\subsection{Universal vs. Existential Quantifiers}

\begin{itemize}
  \item The Universal Quantifier, $\forall$, reads:\\
    $\forall x P(x)$ \\
    for all $x$, $P(x)$ is true

  \item The Existential Quantifier $\exists$

  \item $u$ is the universal discourse
\end{itemize}

\subsection{The Problem}

Free Variables are bad

Bound variables are good

A predicate is a function that maps variables to truth values, allowing one to
go beyond atomic propositions.

Quantifiers are things that allow us to bind variables to a domain.

Predicates are not propositions, unless we replace the free variable with
a logical constant, or bind it's variables with a quantifier.

Going from a generalization to an instance is called instantiation, which is 
very important for acting on quantified predicates using the rules of inference
and the laws of logic.

\subsubsection{P and Q}

Let $P(x)$ and $Q(x)$ be open statements defined for $\mathbf{u}$.

These two statements are logically equivalent, written

\begin{equation*}
  \forall x [P(x) \Leftrightarrow Q(x)]
\end{equation*}

when $P(a) \leftrightarrow Q(a) \forall a \in \mathbf{u}$.

*State the contrapositive, converse and inverse. (exercise)

$\forall x P(x) | x \in \mathbb{R} \implies something$
how to specify the domain of a quantifier

Quantifiers have higher precedence than all logical connectives

Don't mix quantifiers unless they are the same kind

\subsubsection{Arguing with Quantified Statements}

1. $\exists m \in \mathbb{Z}^+ \forall n \in \mathbb{Z}^+ (m \leq n)$

2. $\forall x \in \mathbb{R}^+ \exists y\in{R}^+ (x \leq y)$

\end{document}
