\documentclass[12pt]{article}
\usepackage{amsmath}
\usepackage{amssymb}
\usepackage{amsthm}
\usepackage{amsfonts}
\usepackage{graphicx}
\usepackage{textcomp}
\usepackage{hyperref}
\usepackage{tikz}
\usepackage{enumitem}
\usepackage{mathtools}

\begin{document}

\title{MACM 101 Lecture 1.1}
\author{Alexander Ng}
\date{Friday, September 13, 2024}

\maketitle

\section{Chapter Summary}

\begin{itemize}
  \item Propositional Logic
    \subitem The Language of Propositions
    \subitem Applications
    \subitem Logical Equivalences and Implication
    \subitem The Laws of Propositional Logic

  \item Predicate Logic
    \subitem The Language of Quantifiers
    \subitem Nested Quantifiers

  \item Proofs
    \subitem Rules of Inference
    \subitem Proof Methods
    \subitem Proof Strategy
\end{itemize}

This document covers everything from Rosen 1.0 to 1.3.

\section{Definitions}


\subsection{Deduction/Deductive Logic}

Deduction is the process of deriving a conclusion from a given set of axioms 
or premises. In Logic, we start from the ground (axioms) and work our way up to
the conclusion.

\subsection{Truth Value}

A truth value can be either true or false, but not both. This comes from the
\textit{principium tertii eclusi} of Aristotle.

\subsubsection{True and False}

We will use 0 and 1 to denote true and false, respectively.

\subsubsection{Unknown Truth Value}

The proposition $u$ is \textit{unknown truth value}.

\subsection{Proposition}

A proposition is a declarative sentence (or statement) that possesses truth value.

\subsubsection{Notation}

Lowercase letters denote primitive propositions, and uppercase letters denote
complex propositions.

Primitive propositions are:

\begin{itemize}
\item Propositions that cannot be decomposed into anything simpler
\item $p: 3 + 5 = 8$
\item $q: \text{It is raining}$
\end{itemize}

\subsection{Examples of things that are not propositions}

\begin{itemize}
\item $p:$ Sit down! $\to$ not a proposition because it is not a declarative
\item $q:$ The statement you are reading is now false. $\to$ not a proposition
  because it is a contradiction.
\item $r:$ The number $x$ is an integer. $\to$ not a proposition because it 
  contains an unspecified variable, which means it's truth value cannot be
  definitively determined without additional information.
\end{itemize}

\subsection{Syntactics and Semantics}

Syntatic reasoning is what can be shown. 

Syntax ~= grammar (rules of sentance construction), the structure of propositions

Semantics reasoning is what is true

Semantics ~= meaning (truth value), the truth value/tables of propositions

\subsection{Literals}

A \textit{literal} is either a primitive proposition or its negation (some
textbooks use ~ to denote a literal)

\section{Operator Syntax}

\begin{enumerate}
  \item Negation - $\neg$ \\
    $q:$ it is raining, $\neg q:$ it is not raining\\
  Everything in this list other than $\neg $  is known as a \textit{logical connective}
  \item Conjunction - $\land$ - Logical and\\
    $p \land q:$ it is raining and it is sunny \\
    $p \land \neg q:$ it is raining and it is not sunny
  \item Disjunction - $\lor$ - Inclusive Or\\
    $p \lor q:$ it is raining or it is sunny \\
    $p \lor \neg q:$ it is raining or it is not sunny
  \item Disjunction - $\oplus$ - Exclusive Or\\
    $p \oplus q:$ it is raining xor it is sunny \\
    $p \oplus \neg q:$ it is raining xor it is not sunny\\
    XOR is generally what is meant in english s\entences like "the meal comes
    with either soup or salad"
  \item Implication - $\to$ - "If, then"
  \item Biconditional - $\leftrightarrow$ - "If and only if"
\end{enumerate}

Nobody knows why OR and XOR are both called Disjunction

All propositions formed with logical connectives are called \textit{compound propositions},
as opposed to \textit{primitive propositions}

Compound propositions need not have causal
relations between atomic components (they
can sound nonsensical and still be valid) –
material implication as opposed to causal
implication, which lacks temporal ordering. (straight from the slides, p. 34)

\section{Semantics}

TRUTH TABLES. that's basically all semantics is

End at PDF 1.1 page 37

\end{document}
