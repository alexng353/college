\documentclass[12pt]{article}
\usepackage{amsmath}
\usepackage{amssymb}
\usepackage{amsthm}
\usepackage{amsfonts}
\usepackage{graphicx}
\usepackage{textcomp}
\usepackage{hyperref}
\usepackage{tikz}
\usepackage{enumitem}
\usepackage{mathtools}
\usepackage{float}
\usepackage{cleveref}
\usepackage{hyperref}
\usepackage{csquotes}

\begin{document}

\title{MACM 101 Chapter 1.3 - Logical Identities}
\author{Alexander Ng}
\date{Sunday, September 15, 2024}

\maketitle

This document covers Rosen 1.3, Pearce 1.1 72-xx.

\section*{Summary}

\begin{enumerate}
\item Tautologies, Contradictions and Contingencies
\item Logical Equivalence
  \subitem Important Logical Equivalences
  \subitem Showing Logical Equivalences
\item Logical Implication (not in Rosen)
\item Normal Forms
  \subitem Distributive Normal Form (DNF)
  \subitem Conjunctive Normal Form (CNF)
\end{enumerate}

% reminder to self: add refs to each section

\pagebreak

\section{Tautologies, Contradictions and Contingencies}

A Tautology (\textbf{T}) is a proposition that is always true.

Example: $p \lor \neg p$

A Contradiction (\textbf{F}) is a proposition that is always false.

Example: $p \land \neg p$

\end{document}
