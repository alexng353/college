\documentclass[12pt]{article}
\usepackage{amsmath}
\usepackage{amssymb}
\usepackage{amsthm}
\usepackage{amsfonts}
\usepackage{graphicx}
\usepackage{textcomp}
\usepackage{hyperref}
\usepackage{tikz}
\usepackage{enumitem}
\usepackage{mathtools}
\usepackage{enumitem}

\begin{document}

\title{MATH 240 Lecture 1.5\\Solution Sets of Linear Systems}
\author{Alexander Ng}
\date{September 16, 2024}

\maketitle

Every linear system can be written in the form $Ax=b$, where $A$ is the
coefficient matrix, $x$ is the vector of unknowns, and $b$ is the vector of
constants.

In a linear system $Ax=0$, there is always a solution of the form $x=\mathbf{0}$,
where $\mathbf{0}$ is the zero vector. This is known as the \textbf{trivial
solution}. Other solutions (there may be 0, 1, or many) are called 
\textbf{non-trivial solutions}.

A linear system is said to be \textbf{homogeneous} if it can be written in the
form $Ax=0$. Homogeneous systems are always consistent, becasue of the
aformentioned trivial solution.


\subsubsection*{Theorem}
The solutions of $Ax=0$ can always be written as a 

\begin{equation*}
  \text{span}(v_1, v_2, \dots, v_n)
\end{equation*}

\subsubsection*{Example}

\begin{align*}
  \begin{bmatrix}
    1 & 1 & 2 & 1 & 0 \\
    0 & 1 & 2 & 0 & 0
  \end{bmatrix}
  &=
  \begin{bmatrix}
    1 & 0 & 0 & 1 & 0 \\
    0 & 1 & 2 & 0 & 0
  \end{bmatrix}\\
  &x_1 + x_4 = 0 \\ 
  &x_2 + 2x_3 = 0;
\end{align*}

\begin{align*}
x_1 = -x_4\\
x_2 = -2x_3\\
x_3 = \text{free}\\
x_4 = \text{free}\\
\end{align*}

Let $x_3 = s \in \mathbb{R}$

Let $x_4 = t \in \mathbb{R}$

\[
  \begin{bmatrix}
    x_1 \\ x_2 \\ x_3 \\ x_4
  \end{bmatrix}
  =
  \begin{bmatrix}
    -t \\ -2s \\ s \\ t
  \end{bmatrix}
  =
  s
  \begin{bmatrix}
    0 \\ -2 \\ 1 \\ 0
  \end{bmatrix}
  +
  t
  \begin{bmatrix}
    -1 \\ 0 \\ 0 \\ 1
  \end{bmatrix}
  = \text{span}(v_1, v_2).
\]

\subsubsection*{Theorem}

Consider the linear systems $Ax=b$ and $Ax=0$.

Suppose $Ap=b$ and $Aw=0$. 

$\bullet$ $p+w$ is a solution of $Ax=b$.

Check. $A(p+w) = Ap + Aw = b + 0 = b$.

If $Ax=b$ is consistent, then the solutions of $Ax=0$ can be expressed as

$x = p + (\text{the solutions of } Ax=0) = p+\text{span}(v_1, v_2, \dots, v_n)$

\subsubsection*{Example}

$A\vec{x} = b$

\[
\begin{bmatrix}
  1 & 1
  \end{bmatrix}
  \cdot
  \begin{bmatrix}
    x \\ y
  \end{bmatrix}
=
\begin{bmatrix}
  1
\end{bmatrix}
\]

$x+y = 1$

$x=1-y$

$y=free$

let $y=t$

$x=1-t$

$\vec{x} = 
\begin{bmatrix}
  x \\y 
\end{bmatrix} 
= \begin{bmatrix}
  1-t \\ t
\end{bmatrix}
= \begin{bmatrix}
  1 \\ 0 
\end{bmatrix}
+ t\begin{bmatrix}
  -1 \\ 1
\end{bmatrix}$

$t\begin{bmatrix}
  -1 \\ 1
\end{bmatrix}\text{ is } \text{span}(\begin{bmatrix}
-1 \\ 1
\end{bmatrix})$

find example in lecture notes and solve

\section*{The Geometry of the solutions to $Ax=b$}

Find the picture in lecture notes

Annotations:

\begin{itemize}
\item How can we describe the solutions of $x+y=1$ in terms of the solutions of
  $x+y=0$?
\item When $t=1$, $w=\begin{bmatrix}
    -1 \\ 1
  \end{bmatrix}$
\item when you take any arbitrary solution to $x+y=0$, and add the vector $p$,
  then add the vector $w$, you get the solutions to $x+y=1$
\end{itemize}

\section*{Theorem 6}

Let $Ax=b$ be a linear system with solution $x=p$, that is *if p is a solution
to $Ax=b$, that means $Ap=b$.

Then,

\begin{enumerate}
\item If $Aw=0$, i.e. $w$ is a solution to $Ax=0$ then, $A(p+w)=b$.
\item If $Az=b$, i.e. $z$ is a solution of $Ax=b$, then $A(z-p)=b$. \\
  in other words, you can find a solution of $Ax=0$
\end{enumerate}

\subsection*{Proof of 6.1}

\begin{itemize}
  \item $Aw=0$
  \item $A(p+w)=Ap+Aw=b+0=b$
\end{itemize}

\subsection*{Proof of 6.2}

$A(z-p) \stackrel{?}{=} Az-Ap = b-b = 0$

\subsubsection*{Theorem 5, C1.4}

\begin{enumerate}
  \item $A(u+v)=Au+Av$
  \item $A(c\cdot u)=c\cdot Au$
\end{enumerate}

Proof of $A(z-p) = Az-Ap$

$A(z-p) \stackrel{\text{Prop. }9}{=} A(z+(-1)\cdot p) \stackrel{\text{Th. 5, prop. }1}{=} 
Az+A((-1)\cdot p) \stackrel{\text{Th. 5, prop. }2}{=} Az+(-1)\cdot Ap$


% Fucking disgusting

\subsection*{Example}

what

\begin{align}
  S &=
  \text{span}(
  \begin{bmatrix}
    1 \\0
  \end{bmatrix},
  \begin{bmatrix}
    0 \\ 2
  \end{bmatrix},
  \begin{bmatrix}
    2 \\2
  \end{bmatrix}
  )\\ &= \text{span}(
    \begin{bmatrix}
      1 \\0
    \end{bmatrix},
    \begin{bmatrix}
      0 \\ 2
    \end{bmatrix}
    )\\
    &= \text{span}(
    \begin{bmatrix}
      1 \\ 0
    \end{bmatrix},
    \begin{bmatrix}
      0 \\ 1
    \end{bmatrix}
    )\\
    &= \mathbb{R}^2
\end{align}

Every simplification of the above $span(\dots) \in \mathbb{R}^2$ has two vectors.

Every vector in $\mathbb{R}^2$ can be written as a linear combination of $e_1$ and $e_2$.

\end{document}
