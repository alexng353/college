\documentclass[12pt]{article}
\usepackage{amsmath}
\usepackage{amssymb}
\usepackage{amsthm}
\usepackage{amsfonts}
\usepackage{graphicx}
\usepackage{textcomp}
\usepackage{hyperref}
\usepackage{tikz}
\usepackage{enumitem}
\usepackage{mathtools}

\begin{document}

\title{MATH 240 Lecture 1.4 - The Matrix Vector Product $Ax$}
\author{Alexander Ng}
\date{September 11, 2024}

\maketitle

\section{Matrices}

An \( m \times n \) matrix is a rectangular array of numbers with \( m \) rows
and \( n \) columns.

\subsection{Example}

\begin{minipage}{0.3\textwidth}
  \centering
  \( 2 \times 3 \) matrix
\end{minipage}
\begin{minipage}{0.5\textwidth}
  \begin{equation}
    \begin{bmatrix}
      1 & 1 & 1 \\
      1 & 2 & 3
    \end{bmatrix}
  \end{equation}
\end{minipage}

\subsection{Definition}

Let \( A \) be an \( m \times n \) matrix with columns \( v_{1}, v_{2},\ldots ,
v_{n} \in \mathbb{R}^{n} \). Then the matrix product \( Av_{1} \) is defined as

$$
  A \cdot x = x_{1} \cdot v_{1} + x_{2} \cdot v_{2} + \ldots + x_{n} \cdot v_{n} \in \mathbb{R}^{m}
$$

  \begin{align}
    \begin{bmatrix}
      1 & 1 & 1 \\
      1 & 2 & 3
    \end{bmatrix}
    \cdot
    \begin{bmatrix}
      1 \\
      2 \\
      3
    \end{bmatrix}
    &=
    1 \cdot \begin{bmatrix}
      1 \\
      1
    \end{bmatrix}
    +
    2 \cdot \begin{bmatrix}
      1 \\
      2
    \end{bmatrix}
    +
    3 \cdot \begin{bmatrix}
      1 \\
      3
    \end{bmatrix} \\
    &= \begin{bmatrix} 
      1 \\ 1
    \end{bmatrix} 
    + \begin{bmatrix}
      2 \\ 4
    \end{bmatrix}
    + \begin{bmatrix}
      3 \\ 9
    \end{bmatrix} \\
    &= \begin{bmatrix}
      6 \\ 14
    \end{bmatrix}
  \end{align}

Through some algebraic reasoning, we can show that this matrix product follows:

Letting $v$ be the vector $\begin{pmatrix} x & y & z \end{pmatrix}$, we have

\begin{align}
  \begin{bmatrix}
    a & b & c \\ d & e & f 
  \end{bmatrix}
  \times
  v 
  &=
  \begin{bmatrix}
    a & b & c \\
    d & e & f 
  \end{bmatrix}
  \times
  \begin{bmatrix}
    x \\ y \\ z
  \end{bmatrix} \\
  &= 
  x \cdot \begin{bmatrix} a \\ d \end{bmatrix}
  + y \cdot \begin{bmatrix} b \\ e \end{bmatrix}
  + z \cdot \begin{bmatrix} c \\ f \end{bmatrix} \\
  &=
  \begin{bmatrix}
    ax + by + cz \\
    dx + ey + fz
  \end{bmatrix} \\
  &=
  \begin{bmatrix}
    R_{1} \cdot v \\
    R_{2} \cdot v
  \end{bmatrix}
\end{align}

Where \( \cdot \) denotes the \underline{matrix dot product}.

\subsection{The Matrix Dot Product}

The \underline{matrix dot product} \( u \cdot v \) is defined as

\begin{equation}
  u \cdot v = u_{1} \cdot v_{1} + u_{2} \cdot v_{2} + \ldots + u_{n} \cdot v_{n}
\end{equation}

\subsubsection{Example}

\begin{equation}
  \begin{bmatrix} 1 & 1 \\ 1 & 2 \end{bmatrix}
  \cdot
  \begin{bmatrix} 4 \\ 5 \end{bmatrix} 
  =
  \begin{bmatrix} 
    1 \cdot 4 + 1 \cdot 5 = 9 \\
    1 \cdot 4 + 2 \cdot 5 = 14
  \end{bmatrix} 
\end{equation}

\subsection{Matrix Scalar Multiplication}

If $A$ is an $m \times n$ matrix with rows $r_{1}, r_{2}, \dots, r_{m}$, then

\begin{equation}
  A \cdot x = 
  \begin{bmatrix} 
    r_{1} \cdot x \\
    r_{2} \cdot x \\
    \vdots \\
    r_{m} \cdot x
  \end{bmatrix}
\end{equation}

\subsection{Permutation Matrices}

\subsubsection{Definition}

A \underline{permutation matrix} is a square binary matrix that represents
a permutation of elements. Each row and column of a permutation matrix contains
exactly one $1$, with all other entries being $0$.

\subsubsection{Basis Matrix in $\mathbb{R}^{3}$}

This is the identity matrix in $\mathbb{R}^{3}$ (denoted by $I_{3}$). When you 
multiply any $v \in \mathbb{R}^{3}$ by this matrix, you get the same vector back.

\begin{equation}
  \begin{bmatrix}
    1 & 0 & 0 \\
    0 & 1 & 0 \\
    0 & 0 & 1
  \end{bmatrix}
\end{equation}

Example:

\begin{equation}
  \begin{bmatrix}
    1 & 0 & 0 \\
    0 & 1 & 0 \\
    0 & 0 & 1
  \end{bmatrix}
  \cdot
  \begin{bmatrix}
    x \\ y \\ z
  \end{bmatrix}
  =
  \begin{bmatrix}
    x \\ y \\ z
  \end{bmatrix}
\end{equation}

This is the case because the first row of the matrix, corresponding to the first
entry of the vector, is $\begin{pmatrix} 1 & 0 & 0 \end{pmatrix}$. 
Taking the dot product of a vector with the first row of the matrix returns
the first entry of the vector.

Following, the second row is $\begin{pmatrix} 0 & 1 & 0 \end{pmatrix}$,
and the third row is $\begin{pmatrix} 0 & 0 & 1 \end{pmatrix}$.

Therefore, the first entry of the vector is $x$, the second entry is $y$,
and the third entry is $z$.

\subsubsection{Permuting Rows of a Vector (Example)}

\begin{equation}
  \begin{bmatrix}
    0 & 0 & 1 \\
    0 & 1 & 0 \\
    1 & 0 & 0
  \end{bmatrix}
  \cdot
  \begin{bmatrix}
    x \\ y \\ z
  \end{bmatrix}
  =
  \begin{bmatrix}
    z \\ y \\ x
  \end{bmatrix}
\end{equation}

If you take the vector $\begin{pmatrix} x & y & z \end{pmatrix}$ and multiply 
it by the first row of the matrix according to the matrix dot product, you get

\begin{equation}
  \begin{bmatrix}
    0 & 0 & 1 \\
    0 & 1 & 0 \\
    1 & 0 & 0
  \end{bmatrix}
  \cdot
  \begin{bmatrix}
    x \\ y \\ z
  \end{bmatrix}
  =
  \begin{bmatrix}
    0 \cdot x + 0 \cdot y + 1 \cdot z \\
    \vdots \\
  \end{bmatrix}
  =
  \begin{bmatrix}
    z \\
    \vdots \\
  \end{bmatrix}
\end{equation}

Following, you can see how the second row of the matrix leaves only the $y$ and
the third row leaves only the $x$ components.

\end{document}
