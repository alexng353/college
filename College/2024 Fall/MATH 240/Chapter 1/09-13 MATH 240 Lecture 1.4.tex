\documentclass[12pt]{article}
\usepackage{amsmath}
\usepackage{amssymb}
\usepackage{amsthm}
\usepackage{amsfonts}
\usepackage{graphicx}
\usepackage{textcomp}
\usepackage{hyperref}
\usepackage{tikz}
\usepackage{enumitem}
\usepackage{mathtools}
\usepackage{enumitem}

\begin{document}

\title{MATH 240 Lecture 1.4\\The Matrix Equation $Ax=b$}
\author{Alexander Ng}
\date{September 13, 2024}

\maketitle

**The equation $Ax=b$ is very important

\section*{Review}

Let $A$ be an $m \times n$ matrix.

Let $x$ be a vector of length $n$, with entries $x_{1}, x_{2}, \dots, x_{n}$.

The definition of how to multiply a matrix by a vector is

If $
  A=\begin{bmatrix}
    \vdots & \vdots & \vdots & \vdots \\
    v_{1} & v_{2} & \dots & v_{n} \\
    \vdots & \vdots & \vdots & \vdots \\
  \end{bmatrix}
$ then, $
  Ax = \begin{bmatrix}
    x_1 \cdot v_{1} + x_2 \cdot v_{2} + \dots + x_n \cdot v_{n}
  \end{bmatrix}
$

If $
  A=\begin{bmatrix}
    r_1\\r_2\\ 
    \dots
    \\r_m
  \end{bmatrix}
$ then $
  Ax = \begin{bmatrix}
    r_1 \cdot x\\
    r_2 \cdot x\\
    \dots\\
    r_m \cdot x
  \end{bmatrix}
$

\pagebreak
\section{Theorem 5: Properties of the matrix vector product}

Let $A$ and $B$ be $m \times n$ matrices over $\mathbb{R}$, and 
$u,v \in \mathbb{R}^{n}$ and $c\in\mathbb{R}$

\subsection{The Distributive Law}

\begin{figure}[h!]
  \begin{equation}
    A(u+v) = Au + Av
    \label{eq:1}
  \end{equation}
\end{figure}

The Distributive Law implies that

\begin{align*}
  A(u+v) &= A(u+v) + A(w) \\
         &= Au + Av + Aw \\
\end{align*}

because addition is commutative.

\subsection{The Associative Law}

\begin{equation}
  A(c \cdot v) = c \cdot A \cdot v
\end{equation}

\subsection{The Distributive Law}

\begin{equation}
  (A+B)\cdot u = Au + Bu
\end{equation}

% \quotedblbase{Proofs of the above properties will be tested on the exams.}
Proofs of the above properties will be tested on the exams.

\subsection{Proving Property \ref{eq:1}}

% \begin{align}
%  a &=b \\
%  c &=d
% \end{align}


% \begin{align}
  % A(u+v) &= Au + Av
  % A(u+v) &= A(\begin{bmatrix} u_1 \\ u_2 \\ \vdots \\ u_n \end{bmatrix} 
  % + \begin{bmatrix} v_1 \\ v_2 \\ \vdots \\ v_n \end{bmatrix}) \\

  % &= \begin{bmatrix}
  %   \vdots & \vdots & & \vdots \\
  %   w_{1} & w_{2} & \dots & w_{n} \\
  %   \vdots & \vdots & & \vdots
  %   \end{bmatrix}
  %
    % \times
    %
    % \begin{bmatrix}
    %   u_1 + v_1 \\
    %   \vdots \\
    %   u_n + v_n
    % \end{bmatrix} \\

% \end{align}

\section{Four ways to represent a linear system}

\noindent

1. Standard Form
\begin{align*}
  2x_1 + 1x_2 &= 5 \\
  1x_1 + 3x_3 &= 7 
\end{align*}

2. Augmented Matrix
$$
  \begin{bmatrix}
    2 & 1 & 5 \\
    1 & 3 & 7
  \end{bmatrix}
$$

3. Vector Equation

$x_1 \cdot v_1 + x_2 \cdot v_2 + \dots + x_n \cdot v_n = b$

\begin{equation*}
b=\begin{bmatrix} 5 \\ 7 \end{bmatrix} =
\begin{bmatrix} 2x_1 + 1x_2 \\ 1x_1 + 3x_3 \end{bmatrix} =
\begin{bmatrix} 2x_1 \\ x_1 \end{bmatrix} +
\begin{bmatrix} 1x_2 \\ 0 \end{bmatrix} +
\begin{bmatrix} 0 \\ 3x_3 \end{bmatrix} \\ = 
x_1 \cdot \begin{bmatrix} 2 \\ 1 \end{bmatrix} +
x_2 \cdot \begin{bmatrix} 1 \\ 0 \end{bmatrix} +
x_3 \cdot \begin{bmatrix} 0 \\ 3 \end{bmatrix}
\end{equation*}

4. Consider the following matrix

\begin{align*}
  \begin{bmatrix}
    2 & 1 & 0 \\
    1 & 0 & 3
  \end{bmatrix}
  \cdot
  \begin{bmatrix}
    x_1 \\ x_2 \\ x_3
  \end{bmatrix}
  &=
  \begin{bmatrix}
    2 \cdot x_1 + 1 \cdot x_2 + 0 \cdot x_3 \\
    1 \cdot x_1 + 0 \cdot x_2 + 3 \cdot x_3
  \end{bmatrix}\\
  &=
  \begin{bmatrix}
    5 \\ 7
  \end{bmatrix}
\end{align*}

This is the same as $Ax=b$ where $A$ is the matrix above and $x$ is the vector
of unknowns to be solved for.

Compare this with $ax=b$ (the linear system). $x=\frac{b}{a}$.

*What is the point of having 4 different ways to represent a linear system?*

Method (1) is for building a linear system from scratch.

Method (2), the augmented matrix, is for solving a linear system that has already
been built.

Method (3) is for proofs.

Method (4) is for reasoning and proofs, because it is concise.


\section{Theorems}

Let $A$ be an $m \times n$ matrix with columns $v_{1}, v_{2}, \dots, v_{n}$ and
x=$\begin{bmatrix} x_{1} & x_{2} & \dots & x_{n} \end{bmatrix}$ be a vector in
$\mathbb{R}^{n}$ and $b$ be a vector in $\mathbb{R}^{m}$.

\subsection{Theorem 3}

$Ax=b$, $\begin{bmatrix}A | b\end{bmatrix}$ and 
$\begin{bmatrix} x_1 v_1 + x_2 v_2 + \dots + x_n v_n \end{bmatrix}$ have the same
solution set(s).

\subsection{Theorem 4}
If $B \sim A$ and $B$ is in REF, then the following statements are equivalent.

\begin{enumerate}[label=(\alph*)]
\item The linear system $Ax=b$ has a solution for every choice of $b\in\mathbb{R}^m$.
\item Every $b\in\mathbb{R}^m$ is a linear combination of the columns of $A$.
\item The span of $v_{1}, v_{2}, \dots, v_{n}$ generates $\mathbb{R}^m$.\\
  In other words, every vector in $\mathbb{R}^m$ can be obtained from the span
  of $v_{1}, v_{2}, \dots, v_{n}$.
\item The matrix $B$ has a pivot (position) in every row.\\
  This is our tool for testing if a..c are true.
\end{enumerate}

% What does it mean for the statement to be equivalent? What is the theorem saying?

What this theorem is saying is that these four statements are \textbf{all} either
simultaneously true or simultaneously false.

\subsubsection{Definition (Equivalence)}

Two statements are equivalent if they are simultaneously true or simultaneously
false.

\subsubsection{Example}
\noindent

Is $Span(
\begin{bmatrix}
  1 \\ 1 \\ 1
\end{bmatrix},
\begin{bmatrix}
  1 \\ 2 \\ 3
\end{bmatrix},
\begin{bmatrix}
  1 \\ 0 \\ 1
\end{bmatrix}
) = \mathbb{R}^3$?
\newline

Step 1. Apply (a) and (b). % (Please clean this up) % this became a problem for future me

\[
A = \begin{bmatrix}
  1 & 1 & 1 \\
  1 & 2 & 3 \\
  1 & 0 & 1
\end{bmatrix}
\]

Apply the row operations:

\[
R_2 := R_2 - R_1; \quad R_3 := R_3 - R_1
\]

\[
\begin{bmatrix}
  1 & 1 & 1 \\
  0 & 1 & -1 \\
  0 & 2 & 0
\end{bmatrix}
\]

Next row operation:

\[
R_3 := R_3 - 2R_2
\]

\[
\begin{bmatrix}
  1 & 1 & 1 \\
  0 & 1 & -1 \\
  0 & 0 & 2
\end{bmatrix}
\]

\[
\implies \text{(d) is true.}
\]

We can prove Theorem 4 by showing equivalence between (c) and (d), (b) and (a),
and (a) and (d).

\subsubsection{Proof that (d) and (a) are equivalent}

(d) $\implies$ (a). If (d) is true, then $[A|b] \sim [B|\vdots]$ in REF has a pivot 
position in every row, meaning $Ax=b$ has a solution.

If (d) is false, then $A \sim B$ has at least one row of zeroes.

Consider the matrix 
\[
B = \begin{bmatrix}
  u \\
  0 & 0 & \dots & 0 & 1
\end{bmatrix}.
\]
B is \textbf{inconsistent} because its augmented matrix has $0 = 1$ in the 
bottom row. Since row operations are reversible, the system $B \sim A$. 

\[
\therefore A \text{ is also inconsistent.}
\]

\[
\implies Ax = b \text{ is inconsistent.}
\]

This means that statement (a) is false.

We have shown that if (d) is false, then there exists a vector $b$ such that $Ax = b$
has no solutions, which implies that (a) is false.

\end{document}
