\documentclass[12pt]{article}
\usepackage{amsmath}
\usepackage{amssymb}
\usepackage{amsthm}
\usepackage{amsfonts}
\usepackage{graphicx}
\usepackage{textcomp}
\usepackage{hyperref}
\usepackage{tikz}
\usepackage{enumitem}
\usepackage{mathtools}

\begin{document}

\title{MATH 240 Lecture 1.2 - Row Reduction to Echelon Form}
\author{Alexander Ng}
\date{September 11, 2024}

\maketitle

\section{Row Reduction to Echelon Form}

\subsection{Definition}
A rectangular matrix is in \textbf{row echelon form} (REF) if

\begin{enumerate}
  \item All zero rows are below non-zero rows.
  \item Each leading nonzero entry of a row is in a column to the right of the
  leading entry of the row above it
  \item All entries in a column below a leading nonzero entry are zero
\end{enumerate}

\subsection{Theorem 1}

Every Matrix can be row reduced to REF and to RREF.
(Each) RREF is unique, that is every matrix is row equivalent to *one*
RREF matrix only.

\subsubsection{Example}

\[
  \begin{bmatrix}
    1 & 1 & 0 \\
    0 & 0 & 2
  \end{bmatrix} 
.\] 

This gives us $x+y = 0$ and $0=2$.

If we take $ R_2 := \frac{1}{2} R_2$, we get the RREF matrix
\[
  \begin{bmatrix}
    1 & 1 & 0 \\
    0 & 0 & 1
  \end{bmatrix}
.\]

This gives us $x+y = 0$ and $0=1$.

The leading entries of a row of a matrix in REF are called pivots.

\subsection{Theorem 2}

A linear system is inconsistent (no solutions) iff the rightmost column of
an augmented matrix in REF has a pivot

\subsection{Avoiding Fractions}

Generally, we want to avoid fractions at all costs.

\subsubsection{Example [Avoiding Fractions]}

Given the matrix

\[
  \begin{bmatrix}
    3 & 2 & 1 \\
    1 & 2 & 3
  \end{bmatrix}
\]

We can reduce in two paths.

The first path begins as follows
\[
  \begin{bmatrix}
    3 & 2 & 1 \\
    1 & 2 & 3
  \end{bmatrix}
  \xrightarrow{R_2\coloneqq R_2-\frac{1}{3} R_1}
  \begin{bmatrix}
    3 & 2 & 1 \\
    0 & \frac{4}{3} & 2\frac{2}{3}
  \end{bmatrix}
\]

This leaves us with some nasty fractions. Instead, if we first swap $R_1$ and 
$R_2$, we get

\[
  \begin{align}
    &\begin{bmatrix}
      3 & 2 & 1 \\
      1 & 2 & 3
    \end{bmatrix}
    \xrightarrow{R_1 := R_2, R_2 := R_1}
    \begin{bmatrix}
      3 & 2 & 1 \\
      1 & 2 & 3
    \end{bmatrix}
    \xrightarrow{R_2\coloneqq R_{2}-3R_1}
    \begin{bmatrix}
      1 & 2 & 3 \\
      0 & -4 & -8
    \end{bmatrix}\\
    &\xrightarrow{R_2\coloneqq -\frac{1}{4}R_2}
    \begin{bmatrix}
      1 & 2 & 3 \\
      0 & 1 & 2
    \end{bmatrix}
  \end{align}
\]

and so on.

\end{document}
