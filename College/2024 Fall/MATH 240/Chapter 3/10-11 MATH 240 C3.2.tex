

\documentclass[12pt]{article}
\usepackage{amsmath}
\usepackage{amssymb}
\usepackage{amsthm}
\usepackage{amsfonts}
\usepackage{graphicx}
\usepackage{textcomp}
\usepackage{hyperref}
\usepackage{tikz}
\usepackage{enumitem}
\usepackage{mathtools}
\usepackage{enumitem}
\usepackage{wasysym}

\begin{document}

\renewcommand{\arraystretch}{1.25} % Adjust row spacing
\setlength{\arraycolsep}{12pt} 

\title{MATH 240 Lecture 3.1\\Properties of Determinants}
\author{Alexander Ng}
\date{October 07, 2024}

\maketitle

Let $A$ be an $3 \times 3$ matrix.

\[
  A = \begin{bmatrix}
    a_{11} & a_{12} & a_{13} \\
    a_{21} & a_{22} & a_{23} \\
    a_{31} & a_{32} & a_{33}
    \end{bmatrix}
.\]

The $(1, 2)$ minor of $A$ is

\[
  A_{12} = \begin{bmatrix}
    a_{21} & a_{23} \\
    a_{31} & a_{33}
  \end{bmatrix}
\]

\subsubsection*{Definition}

Let $A$ be an $n \times n$ matrix. The determinant of $A$ is defined as.

$\det(A) = \begin{cases}
  n=1 & a_{11}\\
  n > 1& a_{11} \cdot \det(A_{11}) - a_{12} \cdot \det(A_{12}) + \dots 
  + (-1)^{n-1} \cdot a_{n1} \cdot \det(A_{1n})
\end{cases}$

Let $M$ be the number of multiplications in this formula.

\[
  M = \begin{cases}
    n=1 & 0 \\
    n=2 & 2 \\
    n>2 & n \cdot M(n-1) + n.
  .\end{cases}
.\]

So, the number of multiplications to compute the determinant grows exponentially.

\begin{align}
  M(3) &= 3 \cdot M(2) + 3 = 3 \cdot 2 \cdot 3 = 9\\
  M(4) &= 4 \cdot M(3) + 4 = 4 \cdot 3 \cdot 4 = 40\\
  M(20) &= 4.1 \times 10^{18} \sim \text{ 132 years at } 10^{9} \text{ multiplications per second}
\end{align}

\section{Theorem.}

\begin{enumerate}[label=(\alph*)]
\item If $A \sim B$ via $R_i \gets R_i + s \cdot R_j$, then $\det(B) = \det(A)$.
\item If $A \sim B$ via $R_i \leftrightarrow R_j, i \neq j$, then $\det(B) = -\det(A)$.
\item If $A \sim B$ via $R_i \gets s R_i, s \neq 0$, then $\det(B) = s \cdot \det(A)$.
\end{enumerate}

This means if we use row reduction (Gaussian Elimination),

\[
  A \sim A_n \text{ in REF}
\]

then we can get $\det(A)$ from $\det(A_n)$.

\subsection{Example}

\begin{equation*}
  A=  \begin{bmatrix}
    3 & 2 & 1 \\
    0 & 1 & 1 \\
    1 & 1 & 0
  \end{bmatrix}
\end{equation*}

$R_3 \gets 3R_3$


\begin{equation*}
  B= \begin{bmatrix}
    3 & 2 & 1 \\
    0 & 1 & 1 \\
    3 & 3 & 0
  \end{bmatrix}
\end{equation*}

$R_3 \gets R_3 - R_1$


\begin{equation*}
  C=\begin{bmatrix}
    3 & 2 & 1 \\
    0 & 1 & 1 \\
    0 & 1 & -1
  \end{bmatrix}
\end{equation*}

$R_3 \gets R_3 - R_2$


\begin{equation*}
  D= \begin{bmatrix}
    3 & 2 & 1 \\
    0 & 1 & 1 \\
    0 & 0 & -2
  \end{bmatrix}
\end{equation*}

$A$ in REF. (Upper Triangular)

$\det(B) = 3\det(A)$

$\det(B) = \det(C) = \det(D)$

\begin{align*}
  \det(D) &= \det(D^T)\\
          &= \det(\begin{bmatrix}
    3 & 0 & 0 \\
    2 & 1 & 0 \\
    1 & 1 & -2
  \end{bmatrix})\\
          &= 3 \cdot 1 \cdot (-2) \\
          &= -6
\end{align*}

\begin{align*}
  \det(B) &= -6 = 3 \cdot \det(A) \\
  \det(A) &= \frac{-6}{3} = -2
\end{align*}

Gaussian Elimination does $\sim \frac{1}{3} n^3$ multiplications.

\[
  \begin{cases}
    n=20 & \frac{20^3}{3} = 2666\\
    n=1000 & \frac{1000^3}{3} = 333,333,333 < 10^9 < \text{ 1 second}
  \end{cases}
\]

Proof of $(c)$

\end{document}
