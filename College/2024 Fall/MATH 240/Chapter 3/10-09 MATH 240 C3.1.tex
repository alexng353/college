
\documentclass[12pt]{article}
\usepackage{amsmath}
\usepackage{amssymb}
\usepackage{amsthm}
\usepackage{amsfonts}
\usepackage{graphicx}
\usepackage{textcomp}
\usepackage{hyperref}
\usepackage{tikz}
\usepackage{enumitem}
\usepackage{mathtools}
\usepackage{enumitem}
\usepackage{wasysym}

\begin{document}

\renewcommand{\arraystretch}{1.25} % Adjust row spacing
\setlength{\arraycolsep}{12pt} 

\title{MATH 240 Lecture 3.1\\The Determinant of a Matrix}
\author{Alexander Ng}
\date{October 07, 2024}

\maketitle

\subsection*{Definition - The Determinant of a Matrix}

Let $A$ be an $n \times n$ matrix. The $ij^{th}$ minor of $A$, denoted $A_{ij}$,
is the $(n-1)\times (n-i)$ matrix obtained from $A$ by deleting the $i$th row
and $j$th column.

Example:

\[
  A = \begin{bmatrix}
    a_{11} & a_{12} & a_{13}\\
    a_{21} & a_{22} & a_{23}\\
    a_{31} & a_{32} & a_{33} 
  \end{bmatrix}
\]

\[
  A_{21} = \begin{bmatrix}
    a_{11} & a_{13}\\
    a_{31} & a_{33}
  \end{bmatrix}
.\]

The determinant of $A$, denoted $\det(A)$ or $|A|$ is a number defined by

\subsubsection*{Example}

\[
  \det{
    \begin{bmatrix}
      7 & 9 & 1\\
      1 & 2 & 3 \\
      0 & 5 & 6
    \end{bmatrix}
  }
  =
  7 \cdot \det{
    \begin{bmatrix}
      2 & 3 \\
      5 & 6
    \end{bmatrix}
  }
  -
  9 \cdot \det{
    \begin{bmatrix}
      1 & 3 \\
      0 & 6
    \end{bmatrix}
  }
  +
  1 \cdot \det{
    \begin{bmatrix}
      1 & 2 \\
      0 & 5
    \end{bmatrix}
  }
\]

\[
  7 \cdot (12 - 15) + 9 \cdot (6 - 0) + 1 \cdot (5 - 0)
\]

\[
-21 -54 +5 = -70
.\]

\subsubsection*{Example 2}

\[
  \det{
    \begin{bmatrix}
      7 & 0 & 0 \\
      1 & 2 & 0 \\
      0 & 5 & 6
    \end{bmatrix}
  }
  =
  7 \cdot \det{
    \begin{bmatrix}
      2 & 0 \\
      5 & 6
    \end{bmatrix}
  }
  -
  0 \cdot \det{
    \begin{bmatrix}
      1 & 0 \\
      0 & 6
    \end{bmatrix}
  }
  +
  0 \cdot \det{
    \begin{bmatrix}
      1 & 2 \\
      0 & 5
    \end{bmatrix}
  }
.\]

\[
  7 \cdot (2 \cdot 6 - 0 \cdot 5) - 0 \cdot (1 \cdot 6 - 0 \cdot 5) + 0 \cdot (1 \cdot 2 - 0 \cdot 5)
.\]

\[
  7 \cdot 2 \cdot 6 = 84
.\]

\subsection*{Theorem 2}

\[
  \text{Let } U = \begin{bmatrix} 
    u_{11} & 0 & 0 & \dots & 0 \\
    u_{21} & u_{22} & 0 & \dots & 0 \\
    u_{31} & u_{32} & u_{33} & \dots & 0 \\
    \vdots & \vdots & \vdots & \ddots & \vdots \\
    u_{n1} & u_{n2} & u_{n3} & \dots & u_{nn}
  \end{bmatrix}
.\]

$U$ is known as a \textbf{Lower Triangular Matrix}. When $U$ is a lower
triangular matrix, the determinant of $U$ is the product of the diagonal
elements of $U$.

There may be multiple matrices that have the same determinant.

\subsection*{Corollary}

\[
  \text{Let } A = \begin{bmatrix}
    a_{11} & 0 & 0 & \dots & 0 \\
    0 & a_{22} & 0 & \dots & 0 \\
    0 & 0 & a_{33} & \dots & 0 \\
    \vdots & \vdots & \vdots & \ddots & \vdots \\
    0 & 0 & 0 & \dots & a_{nn}
  \end{bmatrix}
.\]

\begin{equation*}
  \det(A) = a_{11} \cdot a_{22} \cdot a_{33} \cdot \dots \cdot a_{nn}
\end{equation*}

\begin{equation*}
  \det(I) = 1
\end{equation*}

\section*{Def.}

Let $A$ be an $n \times n$ matrix. The $ij^{th}$ cofactor of $A$ is 
$C_{ij} = (-1)^{i+j} \det(A_{ij})$.

\subsection*{Theorem 1}

$\det(A)$ can be computed by expanding along row $i$ or down col. $j$. as follows:

\begin{equation*}
  \det(A) = a_{i1} C_{i1} + a_{i2} C_{i2} + \dots + a_{in} C_{in} \to \text{row i}
\end{equation*}

\begin{equation*}
  \det(A) = a_{1j} C_{1j} + a_{2j} C_{2j} + \dots + a_{nj} C_{nj} \to \text{col j}
\end{equation*}

\subsubsection*{Example}

\[
  \text{Let } A = \begin{bmatrix}
    7 & 9 & 1\\
    1 & 2 & 3 \\
    0 & 5 & 6
  \end{bmatrix}
.\]

We should choose the first column to expand along since there is a $0$ entry and
a $1$ entry in the first column.

\begin{equation*}
  \det(A) = a_{11} \cdot (-1)^{1+1} \cdot \det(A_{11}) 
  + a_{21} \cdot (-1)^{1+2} \cdot \det(A_{21}) 
  + a_{31} \cdot (-1)^{1+3} \cdot \det(A_{31})
\end{equation*}

\begin{equation*}
  = 7 \cdot 1 \cdot \det(\begin{bmatrix}
    2 & 3 \\
    5 & 6
  \end{bmatrix})
  + 1 \cdot (-1) \cdot \det(\begin{bmatrix}
    9 & 1 \\
    5 & 6
  \end{bmatrix})
  + \mathbf{0}
\end{equation*}

\begin{equation*}
  = 7 \cdot (12 - 15) + (54 - 5) + 0 = -70
\end{equation*}

\subsection*{Properties of the Determinant}

\subsubsection*{Theorem 4 (Gun #13 IMT)}

$A$ is invertible $\iff \det(A) \neq 0$

\subsubsection*{Theorem 5}

$\det(A^T) = \det(A)$

\subsubsection*{Theorem 6}

$\det(AB) = \det(A) \cdot \det(B)$

\subsubsection*{Determinant of the Sum of Matrices}

Is $\det(A+B) = \det(A) + \det(B)$? No!

\[
  I = \begin{bmatrix}
    1 & 0\\
    0 & 1
  \end{bmatrix}
  = \begin{bmatrix}
    1 & 0 \\
    0 & 0
  \end{bmatrix}
  + \begin{bmatrix}
    0 & 0 \\
    0 & 1
  \end{bmatrix}
\]

Clearly, $\det(I) = 1$. However, $\det(A+B) = \det(A) + \det(B)$ is false by
contradiction, since any matrix with a $0$ row or column has $\det(A) = 0$.

\[
 1 \neq 0 + 0
.\]

\subsubsection*{Proof of Theorem 5}



\end{document}
