\documentclass[12pt]{article}
\usepackage{amsmath}
\usepackage{amssymb}
\usepackage{amsthm}
\usepackage{amsfonts}
\usepackage{graphicx}
\usepackage{textcomp}
\usepackage{hyperref}
\usepackage{tikz}
\usepackage{enumitem}
\usepackage{mathtools}
\usepackage{enumitem}
\usepackage{wasysym}

\begin{document}

\renewcommand{\arraystretch}{1.25} % Adjust row spacing
\setlength{\arraycolsep}{12pt} 

\title{MATH 240 Lecture 2.2\\The Inverse $A^{-1}$ of a matrix $A$}
\author{Alexander Ng}
\date{October 07, 2024}

\maketitle

Let $A$ be an $n \times n$ matrix. $A$ is invertible if there is an $n \times n$
matrix $A^{-1}$, called the inverse of $A$, such that $AA^{-1} = A^{-1}A = I_n$.

\textbf{Theorem 7 - C2.2} $A$ is invertible $\iff A \sim I_n$.

($\implies$) If $A$ is invertible, then $A \sim I_n$.

($\impliedby$) If $A \sim I_n$, then $A$ is invertible.

\subsection*{Proof of Theorem 7-2.2}

Suppose $A$ is invertible. Then $A^{-1}$ exists. Let $A \sim B$ where $B$ is
in REF.
\indent (Th5-2.2) If $A$ is invertible, then $Ax=b$ has a unique solution.
\indent (Th4-1.4) $B$ in REF has a pivot position in every row

We know $A$ is square $\implies B = \begin{bmatrix}p_1 & x & x \\ 0 & p_2 & x \\ 0 & 0 & p_3 \end{bmatrix}$
where $p$ are pivots. We can then apply row operations to $B$ to get $B'$ in RREF. 
$B'$ in RREF = $I_n$

\subsection*{Elementary Matrices}
Every elementary row operation corresponds to a multiplication of $A$ by an 
$n \times n$ elemtary matrix $E$, i.e. if $A \sim B$, then $B = EA$. Since row 
operations are invertible, $E$ is invertible.

\subsubsection*{Example}

For $n=3$ and $R_2 \gets R_2 - 2R_1$, find $E$ and $E^{-1}$.

Start with the $I_n$, $I_3$.

\[
  I_3 =
  \begin{bmatrix}
    1 & 0 & 0 \\
    0 & 1 & 0 \\
    0 & 0 & 1
  \end{bmatrix}
\]

Apply $R_2 \gets R_2 - 2R_1$ to $I_3$.

\[
  E = 
  \begin{bmatrix}
    1 & 0 & 0 \\
    -2 & 1 & 0 \\
    0 & 0 & 1
  \end{bmatrix}
\]

Check.

\[
  EA =
  \begin{bmatrix}
    1 & 0 & 0 \\
    -2 & 1 & 0 \\
    0 & 0 & 1
  \end{bmatrix}
  \cdot
  \begin{bmatrix}
    a & b & c \\
    u & v & w \\
    x & y & z
  \end{bmatrix}
\]

\[
  \begin{bmatrix}
    a & b & c \\
    -2a+u & -2b+v & -2c+w \\
    x & y & z
  \end{bmatrix}
\]

Find the inverse.

\[
  I_3 =
  \begin{bmatrix}
    1 & 0 & 0 \\
    0 & 1 & 0 \\
    0 & 0 & 1
  \end{bmatrix}
\]

Apply the \textbf{Inverse} of the row operation to $I_3$.

$R_2 \gets R_2 + 2R_1$

\[
  E^{-1} =
  \begin{bmatrix}
    1 & 0 & 0 \\
    2 & 1 & 0 \\
    0 & 0 & 1
  \end{bmatrix}
\]

% HOMEWORK:

Check $E E^{-1} = I_3$.

Check $E^{-1} E = I_3$.


\subsection*{Proof of Part 2 of Theorem 7-2.2}

Suppose $A\sim I$. \qquad $A \sim A_1 \sim A_2 \dots A_{m-1} \sim I$

$= E_1 A_1 \dots E_m A_m = I$

Let $E_1 = E_2 \dots E_m$ be the elementary matrices corresponding to the row 
operations in $A \sim I$.

So,

\begin{align*}
  A_1 &= E_1 \cdot A \\
  A_2 &= E_2 \cdot A_1 \\
  \vdots \\
  I &= E_m \cdot A_{m-1}
\end{align*}

Because $A_2 = E_2 \cdot A_1$, we can write $A_2 = E_2 \cdot (E_1 \cdot A)$.

\[
  I = E_m(\dots(E_2 \cdot (E_1 \cdot A))) = (E_m \dots E_2 \cdot E_1) \cdot A
.\]

We can call the matrix product $E_m \dots E_2 \cdot E_1$ the equivalent matrix $C$.

We now have $CA=I$. Is $AC=I$?

We know that $E_1, E_2, \dots, E_m$ are invertible.

\indent\textbf{Theorem 6b-2.2} $(ABC)^{-1} = C^{-1}B^{-1}A^{-1}$

$C^{-1} = E_1^{-1} \cdot E_2^{-1} \dots E_m^{-1}$

$CA = I$

$C^{-1}(CA) = C^{-1}I$

$(C^{-1}C) A = C^{-1}I \implies A = C^{-1} \implies AC = C^{-1}C = I$

\section*{Invertible Linear Transformations}

Suppose $T: \mathbb{R}^n \to \mathbb{R}^n$ is a linear transformation.
T is invertible if there exists another linear transformation 
$S: \mathbb{R}^n \to \mathbb{R}^n$ such that.

\begin{enumerate}
  \item $S(T(x)) = x$
  \item $T(S(y)) = y$
\end{enumerate}

\subsection*{Theorem 9}

Let $T: \mathbb{R}^n \to \mathbb{R}^n$ be a linear transformation with standard
matrix $A$ so $T(x) = Ax$.

$T$ is invertible $\iff A$ is invertible.

\subsubsection*{Proof of Theorem 9}

\noindent

\textbf{Part 1}

$(\impliedby)$ If $T$ is invertible, then $A$ is invertible.

Assume $A$ is invertible. So $A^{-1}$ exists.

Consider $S(y) = A^{-1}y$. Then, 

\begin{equation}
  S(T(X)) = S(Ax) = A^{-1}(Ax) = (A^{-1}A)x = Ix = x
\end{equation}

\begin{equation}
  T(S(X)) = T(A^{-1}x) =  A(A^{-1}x) = (A^{-1}A)x = Ix = x
\end{equation}

\textbf{Part 2}

$(\implies)$ If $A$ is invertible, then $T$ is invertible.

Assume $T(x) = Ax$ is invertible.

Let $S$ be the inverse of $T$. S is a Linear Transformation. So, let $S(y) = Cy$,
where $C$ is the standard matrix of $S$.

$S(T(x)) = x \implies C(Ax) = x \implies (CA) x = x. \forall x \in \mathbb{R}^n$

$\implies (CA)e_i=e_i$

\[
  CA = 
  (CA)
  \begin{bmatrix}
    \vdots & \vdots & & \vdots\\
    e_i & e_2 & \dots & e_n\\
    \vdots & \vdots & & \vdots
  \end{bmatrix}
  = \begin{bmatrix}
    (CA)e_1 & (CA)e_2 & \dots & (CA)e_n
  \end{bmatrix} 
\]


\[
  = \begin{bmatrix}
    e_1 & e_2 & \dots & e_n
  \end{bmatrix}
  = I_n
\]

We have $CA=I_n$. Is $AC=I_n$?

Yes, by IMT.

\end{document}

