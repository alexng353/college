
\documentclass[12pt]{article}
\usepackage{amsmath}
\usepackage{amssymb}
\usepackage{amsthm}
\usepackage{amsfonts}
\usepackage{graphicx}
\usepackage{textcomp}
\usepackage{hyperref}
\usepackage{tikz}
\usepackage{enumitem}
\usepackage{mathtools}
\usepackage{enumitem}
\usepackage{wasysym}

\begin{document}

\renewcommand{\arraystretch}{1.25} % Adjust row spacing
\setlength{\arraycolsep}{12pt} 

\title{MATH 240 Lecture 2.2\\The Inverse $A^{-1}$ of a matrix $A$}
\author{Alexander Ng}
\date{October 04, 2024}

\maketitle

$A+B$, $A-B$, $r \cdot A$, $A\cdot B$, $A^T$, $A^{-1}$

Let $I_n$ be the $n \times n$ identity matrix. (Matrix property 5);

\[
  \begin{bmatrix}
    1 & ... & 0 \\
    ... & 1 & ... \\
    0 & ... & 1
  \end{bmatrix}
\]

$A \cdot I_n = A$ and $I_n \cdot A = A$. (Matrix multiplication by the Identity
Matrix is commutative).

\section{Definition of the Inverse}

Let $A$ be an $n \times n$ matrix. (The definition of the inverse of a matrix is
only for square matrices.)

$A$ is \textbf{invertible} or \textbf{nonsingular} if there exists an $n \times n$
matrix $C$ such that $AC = CA = I_n$.

$C$ is unique. (Theorem)

\subsection{Proof of the Uniqueness of $C$}

Suppose $C$ and $B$ are inverses of $A$.

\[
  AC = I_n \text{ and } AB = I_n \text{ and } CA = I_n \text{ and } CB = I_n
\]

Using the fact that $AC = I_n$, substitute in $AC$ as $I_n$ in $B = B \cdot I_n$.

$B = B \cdot I_n = B(AC) = (BA)C = I_n \cdot C = C$

$\qed$

\section{Forgot Name}

In $\mathbb{R}$, 

\begin{align*}
  ax &= b \\
  x &= b/a \\
  x &= a^{-1}b \{ a \neq 0 \}
\end{align*}

in $\mathbb{R}^n$,

\begin{align*}
  Ax &= b \\
  x &= A^{-1}b \{ A \text{ is invertible} \}
\end{align*}

If $A$ is an invertible matrix, then the linear system $Ax=b$ has the unique
solution $x = A^{-1}b$.

\subsection{Proof}

\begin{align*}
  Ax &= b \\
  \text{Is } A(A^{-1}b) &= b \\
  \\
  A(A^{-1}b) &= b \\
             &= (A A^{-1})b \\
             &= b 
\end{align*}

Suppose $Au = b$ and $Av = b$. $\implies Au = Av$.

$A$ is invertible $\implies A^{-1} \implies A^{-1} (Au) = A^{-1} (Av) = b$.

$A^{-1} (Au) = (A^{-1}A)u = u$.

$A^{-1} (Av) = (A^{-1}A)v = v$.

$\therefore u = v$.

$\qed$

\section{Theorem 6 (Properties of $A^{-1}$)}

\begin{enumerate}[label=(\alph*)]
  \item If $A$ is invertible, then $A^{-1}$ is also invertible and $(A^{-1})^{-1} = A$.
  \item If $A$ and $B$ are invertible, then $(AB)^{-1} = B^{-1}A^{-1}$.
  \item If $A$ is invertible, $A^T$ is also invertible and $(A^T)^{-1} = (A^{-1})^T$.
\end{enumerate}

\subsection{Proof of (b)}

Apply the Associative Law over and over again.

$(AB)(B^{-1}A^{-1}) = A(B(B^{-1}A^{-1})) = A((BB^{-1})A^{-1}) = A(I_nA^{-1}) 
= A(A^{-1}) = I_n$.

Prove $(B^{-1}A^{-1}) (AB) = I_n$. (do it at home)

\subsection{Exercise 2.2-25}

Suppose $A$, $B$ and $C$ are invertible. So $A^{-1}$, $B^{-1}$ and $C^{-1}$ exist.

Find $A\cdot B\cdotC)^{-1} = ((AB)\cdot C)^{-1}$.

\section{How to find the inverse of a matrix}

Is $A = \begin{bmatrix} 0 & 1 \\ -1 & 0 \end{bmatrix}$ invertible? If so, find
$A^{-1}$.

Let's find $C$ such that $AC=I_n$. Then check $CA=I_n$.

\[
  \begin{bmatrix}
    0 & 1 \\
    -1 & 0
  \end{bmatrix}
  \begin{bmatrix}
    a & b \\
    c & d
  \end{bmatrix}
  =
  \begin{bmatrix}
    1 & 0 \\
    0 & 1
  \end{bmatrix}
\]

\[
  \begin{bmatrix}
    c & d \\
    -a & -b
  \end{bmatrix}
  =
  \begin{bmatrix}
    1 & 0 \\
    0 & 1
  \end{bmatrix}
\]

\[
  \begin{bmatrix}
    c = 1, d = 0 \\
    a = 0, b = 1
  \end{bmatrix}
  \implies
  C =
  \begin{bmatrix}
    0 & -1 \\
    1 & 0
  \end{bmatrix}
\]

Check (at home).

\subsection{Theorem}

Let $A = \begin{bmatrix} a & b \\ c & d \end{bmatrix}$. If $ad-bc \neq 0$, then
$A$ is invertible.

And $A^{-1} = \frac{1}{ad-bc} \begin{bmatrix} d & -b \\ -c & a \end{bmatrix}$.

The thingie $ad-bc$ is called the \textbf{determinant} of $A$. (This will be
part of Ch.3)

\subsection{Theorem 7 (The most beautiful algorithm)}

An $n\times n$ matrix $A$ is invertible $\iff A \sim I_n$.

? Calculate $A^{-1}$. Let's find $C$ such that $AC=I_n$. Then check $CA=I_n$.

Calculate the inverse of $A=\begin{bmatrix} 1 & 1 \\ 2 & 1 \end{bmatrix}$.

\[
  \left[
    \begin{array}{cc|cc}
      1 & 1 & 1 & 0 \\
      2 & 1 & 0 & 1 \\
    \end{array}
  \right]
\]

$R_2 \to R_2 - 2R_1$

\[
  \left[
    \begin{array}{cc|cc}
      1 & 1 & 1 & 0 \\
      0 & -1 & -2 & 1 \\
    \end{array}
  \right]
\]

$R_1 \to R_1 + R_2$

$R_2 \to -R_2$

\[
  \left[
    \begin{array}{cc|cc}
      1 & 0 & -1 & 1 \\
      0 & 1 & 2 & -1 \\
    \end{array}
  \right]
\]

Calculate $A \cdot A^{-1}$ (at home).

\end{document}
