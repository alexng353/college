\documentclass[12pt]{article}
\usepackage{amsmath}
\usepackage{amssymb}
\usepackage{amsthm}
\usepackage{amsfonts}
\usepackage{graphicx}
\usepackage{textcomp}
\usepackage{hyperref}
\usepackage{tikz}
\usepackage{enumitem}
\usepackage{mathtools}
\usepackage{enumitem}
\usepackage{wasysym}

\begin{document}

\title{MATH 240 Lecture 1.7\\Matrix Operations}
\author{Alexander Ng}
\date{September 18, 2024}

\maketitle

\section{Matrix Operations}

Let $A$ be an $m \times n$ matrix, where $m$ is the number of rows and $n$ is
the number of columns. The notation $a_{ij}$ refers to the entry of the matrix $A$ in row $i$ and 
column $j$.

\begin{equation*}
  A = \begin{bmatrix}
    a_{11} & a_{12} & \dots & a_{1n} \\
    a_{21} & a_{22} & \dots & a_{2n} \\
    \vdots & \vdots & a_{ij} & \vdots \\
    a_{m1} & a_{m2} & \dots & a_{mn}
  \end{bmatrix}
\end{equation*}

\subsection{Definitions}

Let $A$, $B$ be $m \times n$ matrices over $\mathbb{R}$, and $r \in \mathbb{R}$.

Define

Matrix addition:

\begin{equation*}
  C = A + B \text{ is defined as } c_{ij} = a_{ij} + b_{ij}
\end{equation*}

Matrix subtraction:

\begin{equation*}
  D = A - B \text{ is defined as } d_{ij} = a_{ij} - b_{ij}
\end{equation*}

Scalar multiplication:

\begin{equation*}
  E = rA \text{ is defined as } e_{ij} = r \cdot a_{ij}
\end{equation*}

In each of these cases, you just apply the corresponding operation to each entry
component-wise.

\subsubsection{Properties}

Properties of matrix addition and scalar multiplication.

These are the same as the properties of vector addition and scalar multiplication.

**Matrix addition is identical to vector addition

\subsection{Matrix Multiplication $C = A \cdot B$}

Definition:

Let $B = \begin{bmatrix}
  \vdots & \vdots & \vdots & \vdots \\
  b_{1} & b_{2} & \dots & b_{n} \\
  \vdots & \vdots & \vdots & \vdots
\end{bmatrix}$. 
$A \cdot B = \begin{bmatrix} Ab_1 & Ab_2 & \dots & Ab_n\end{bmatrix}$

Example:

\begin{equation*}
  \begin{bmatrix}
    1 & 1 \\ 
    2 & 2
  \end{bmatrix}
  \begin{bmatrix}
    3 & 4 \\ 
    5 & 6
  \end{bmatrix}
  =
  \begin{bmatrix}
    8 & 10 \\ 
    16 & 20
  \end{bmatrix}
\end{equation*}

\begin{equation*}
  \begin{bmatrix}
    1 & 1 \\ 
    2 & 2
  \end{bmatrix}
  \cdot
  \begin{bmatrix}
    3 \\ 5
  \end{bmatrix}
  =
  \begin{bmatrix}
    3 + 5 = 8 \\
    6 + 10 = 16
  \end{bmatrix}
\end{equation*}

\begin{equation*}
  \begin{bmatrix}
    1 & 1 \\ 
    2 & 2
  \end{bmatrix}
  \cdot
  \begin{bmatrix}
    4 \\ 6
  \end{bmatrix}
  =
  \begin{bmatrix}
    4 + 6 = 10 \\
    8 + 12 = 20
  \end{bmatrix}
\end{equation*}

\subsubsection{Shortcut: The Row Column Rule for computing $A \cdot B$}

For any $C = A \cdot B$, we can shortcut the computation by computing the
dot product of each row of $A$ with it's corresponding column of $B$.

Notice, in general, $A \cdot B \ne B \cdot A$. Matrix multiplication is not
commutative.

\pagebreak

The matrices don't have to be square.

\subsubsection{Example}

Let $A = \begin{bmatrix}
  1 & 1 & 1 \\
  2 & 2 & 2
\end{bmatrix}$


Let $B = \begin{bmatrix}
  1 & 1 \\
  2 & 1 \\
  3 & 1
\end{bmatrix}$

In order to compute $A \cdot B$, the number of columns of $A$ must equal the
number of rows of $B$. $A$ must be an $m \times n$ matrix, and $B$ must be an
$n \times p$ matrix. $C$ becomes an $m \times p$ matrix.

Suppose we have two linear transformations $T_1(x) = Ax$ and $T_2(x) = Bx$.

Let $T_3(x) = T_1(T_2(x)) = T_1(Bx) = ABx$.

Is $T_3(x)$ also a linear transformation? Yes.

Check: (The following properties must hold for $T_3(x)$ to be a linear
transformation)

1. $T_3(x+y) = T_3(x) + T_3(y)$

% clean me up please
$A(B(x+y)) = A(Bx + By) = A(Bx) + A(By) = T_1(T_2(x)) + T_1(T_2(y)) = T_3(x) + T_3(y)$

2. $T_3(cx) = cT_3(x)$

$T_3$'s standard matrix is going to be $AB$.

\textbf{Thm.} $A(Bx) = (AB)x$

\end{document}
